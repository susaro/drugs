%\subsection{Minimal rebinding effect: Optimization problem in terms of eigenvectors (reversible)}

%\section{\fontsize{12}{15}\selectfont Introduction}
\subsection*{A.1. Optimization problem in terms of eigenvectors: reversible system}

\begin{figure}[h]
	%\label{fig:reb_example}
	\centering
	\subfigure[The minimal rebinding effect compared to the degree of non-reversibility of the clustered system $Q_c$.]{\includegraphics[width=0.46\textwidth]{figures/rebinding/200_random/reb_opt_nonrev_200_random.eps}}
	\subfigure[The minimal and the real rebinding effect compared to the degree of non-reversibility of the clustered system $Q_c$.]{\includegraphics[width=0.46\textwidth]{figures/rebinding/200_random/reb_nonrev_200_random.eps}}
	%, increasing the probability of a fast \textbf{rebinding}.
	\hspace{20pt}
	\subfigure[The minimal rebinding effect $\det(\Sopt)$ compared to the real rebinding effect $\det(\Sreal)$ included in $Q_c$.]{\includegraphics[width=0.46\textwidth]{figures/rebinding/200_random/reb_200_random.eps}}
	\caption{The system described by the transition matrix $P$ is clustered with $200$ randomly generated feasible transformation matrices $A$.} % in order to compare some important parameters
\end{figure}
\newpage

\subsection*{A.2. Optimization problem in terms of Schur vectors: reversible}
\begin{figure}[h]
%	\label{fig:reb_example_nonrev}
	\centering
	\subfigure[The minimal rebinding effect compared to the degree of non-reversibility of the clustered system $Q_c$.]{\includegraphics[width=0.46\textwidth]{figures/rebinding/200_random_schur/reb_opt_nonrev_200_random.eps}}
	\subfigure[The minimal and the real rebinding effect compared to the degree of non-reversibility of the clustered system $Q_c$.]{\includegraphics[width=0.46\textwidth]{figures/rebinding/200_random_schur/reb_nonrev_200_random.eps}}
	%, increasing the probability of a fast \textbf{rebinding}.
	\hspace{20pt}
	\subfigure[The minimal rebinding effect $\det(\Sopt)$ compared to the real rebinding effect $\det(\Sreal)$ included in $Q_c$.]{\includegraphics[width=0.46\textwidth]{figures/rebinding/200_random_schur/reb_200_random.eps}}
	\caption{The system described by the transition matrix $P$ is clustered with $200$ randomly generated feasible transformation matrices $A$ for the values $\epsilon = 0$, $\delta = 0$.} % in order to compare some important parameters
\end{figure}
\newpage

\subsection*{A.3. Optimization problem in terms of Schur vectors: non-reversible, real eigenvectors}
\begin{figure}[h]
	%	\label{fig:reb_example_nonrev}
	\centering
	\subfigure[The minimal rebinding effect compared to the degree of non-reversibility of the clustered system $Q_c$.]{\includegraphics[width=0.46\textwidth]{figures/rebinding/200_random_schur/epsilon0.004/reb_opt_nonrev_200_random.eps}}
	\subfigure[The minimal and the real rebinding effect compared to the degree of non-reversibility of the clustered system $Q_c$.]{\includegraphics[width=0.46\textwidth]{figures/rebinding/200_random_schur/epsilon0.004/reb_nonrev_200_random.eps}}
	%, increasing the probability of a fast \textbf{rebinding}.
	\hspace{20pt}
	\subfigure[The minimal rebinding effect $\det(\Sopt)$ compared to the real rebinding effect $\det(\Sreal)$ included in $Q_c$.]{\includegraphics[width=0.46\textwidth]{figures/rebinding/200_random_schur/epsilon0.004/reb_200_random.eps}}
	\caption{The system described by the transition matrix $P$ is clustered with $200$ randomly generated feasible transformation matrices $A$ for the values $\epsilon = 0.004$, $\delta = 0$.} % in order to compare some important parameters
\end{figure}
\newpage

%\subsection*{Minimal rebinding effect: Optimization problem in terms of Schur vectors (non-reversible, complex eigenvectors)}
%\begin{figure}[h]
	%	\label{fig:reb_example_nonrev}
%	\centering
%	\subfigure[The minimal rebinding effect compared to the degree of non-reversibility of the clustered system $Q_c$.]{\includegraphics[width=0.46\textwidth]{figures/rebinding/200_random_schur/gamma0.01/reb_opt_nonrev_200_random.eps}}
%	\subfigure[The minimal and the real rebinding effect compared to the degree of non-reversibility of the clustered system $Q_c$.]{\includegraphics[width=0.46\textwidth]{figures/rebinding/200_random_schur/gamma0.01/reb_nonrev_200_random.eps}}
	%, increasing the probability of a fast \textbf{rebinding}.
%	\hspace{20pt}
%	\subfigure[The minimal rebinding effect $\det(\Sopt)$ compared to the real rebinding effect $\det(\Sreal)$ included in $Q_c$.]{\includegraphics[width=0.46\textwidth]{figures/rebinding/200_random_schur/gamma0.01/reb_200_random.eps}}
%	\caption{The system described by the transition matrix $P$ is clustered with $200$ randomly generated feasible transformation matrices $A$ for the values $\epsilon = 0$, $\delta = 0.01$.} % in order to compare some important parameters
%\end{figure}
%\newpage

\subsection*{A.4. Optimization problem in terms of Schur vectors: non-reversible, non-diagonalizable}
\begin{figure}[h]
	%	\label{fig:reb_example_nonrev}
	\centering
	\subfigure[The minimal rebinding effect compared to the degree of non-reversibility of the clustered system $Q_c$.]{\includegraphics[width=0.46\textwidth]{figures/rebinding/200_random_schur/epsilon0.004delta0.01/reb_opt_nonrev_200_random.eps}}
	\subfigure[The minimal and the real rebinding effect compared to the degree of non-reversibility of the clustered system $Q_c$.]{\includegraphics[width=0.46\textwidth]{figures/rebinding/200_random_schur/epsilon0.004delta0.01/reb_nonrev_200_random.eps}}
	%, increasing the probability of a fast \textbf{rebinding}.
	\hspace{20pt}
	\subfigure[The minimal rebinding effect $\det(\Sopt)$ compared to the real rebinding effect $\det(\Sreal)$ included in $Q_c$.]{\includegraphics[width=0.46\textwidth]{figures/rebinding/200_random_schur/epsilon0.004delta0.01/reb_200_random.eps}}
	\caption{The system described by the transition matrix $P$ is clustered with $200$ randomly generated feasible transformation matrices $A$ for the values $\epsilon = 0.004$, $\delta = 0.01$.} % in order to compare some important parameters
\end{figure}
\newpage

\subsection*{A.5. Optimization problem in terms of Schur vectors: non-reversible, complex eigenvectors}
\begin{figure}[h]
	%	\label{fig:reb_example_nonrev}
	\centering
	\subfigure[The minimal rebinding effect compared to the degree of non-reversibility of the clustered system $Q_c$.]{\includegraphics[width=0.46\textwidth]{figures/rebinding/200_random_schur/epsilon0.002gamma0.01/reb_opt_nonrev_200_random.eps}}
	\subfigure[The minimal and the real rebinding effect compared to the degree of non-reversibility of the clustered system $Q_c$.]{\includegraphics[width=0.46\textwidth]{figures/rebinding/200_random_schur/epsilon0.002gamma0.01/reb_nonrev_200_random.eps}}
	%, increasing the probability of a fast \textbf{rebinding}.
	\hspace{20pt}
	\subfigure[The minimal rebinding effect $\det(\Sopt)$ compared to the real rebinding effect $\det(\Sreal)$ included in $Q_c$.]{\includegraphics[width=0.46\textwidth]{figures/rebinding/200_random_schur/epsilon0.002gamma0.01/reb_200_random.eps}}
	\caption{The system described by the transition matrix $P$ is clustered with $200$ randomly generated feasible transformation matrices $A$ for the values $\epsilon = 0.002$, $\gamma = 0.01$.} % in order to compare some important parameters
\end{figure}