\section{Galerkin Projection}
%interpret $P_Q$ as a projected transfer operator
\label{sec:galerkin}

%Until now
%high-dimensional Markov processes
So far we considered Markov processes on very large, possibly continuous, state spaces.
For many applications, simulations of a given process are needed in order to obtain informations about the corresponding system. But computations on \textbf{large} state spaces require an enormous amount of computation power and time. With larger state spaces, the computation effort increases exponentially fast, see ``curse of dimension''. \marginpar{see ..}
%To avoid this
%In order to reduce the computation effort,
Therefore, we are interested in reducing the number of states in order to make computations more feasible.
%practicable
Such a reducing can for instance be done by ``grouping similar states together''. In this section, we will develop a mathematical concept/tool which enables us to create such a reduced model.

%which shall inherit the most important properties of our original process. This can be done by a Galerkin projection/discretization.

\subsubsection*{Galerkin Projection}
%for Galerkin projection (ansatz space) and/or/= membership functions to apply PCCA+
%At first, we need to chose an appropriate/ desired/ favoured/ convenient ansatz space
%in order to design...
The first step in order to create our desired finite process is to determine a convenient state space $D \subset L^2(\mu)$.
%Instead of choosing a basis that consists of characteristic functions
%adopt/select; approach/idea
Instead of just chosing characteristic functions as the basis of our new state space, we will adopt the concept of a partition of unity. \marginpar{real?}
Using that more general idea gives us more possibilities/options in/for later applications.
%flexibility

\begin{defi}[Partition of Unity]
A family of measurable functions $\{ \cfam \} \subset L^2(\mu)$ \marginpar{why L2?} is called a \textit{partition of unity} if the following two conditions are fullfilled:
\begin{enumerate}
\item The $\chi_i$ are non-negative and  linear independent
%pairwise indep?
\item $\sum_{i=1}^n \chi_i(q) = 1$ for all $q \in \X$ \marginpar{a.e.??}
\end{enumerate}
\end{defi}

\begin{defi}[Galerkin Projection]
Let $\{ \cfam \}$ be a partition of unity, $D = \mathrm{span}\{\cfam\}$ the associated finite-dimensional ansatz space and $\hat{S} \in \R^{n\times n}$ with $\hat{S}_{kj} = \langle \chi_k, \chi_j \rangle_\mu$. The \textit{Galerkin projection} onto $D$ is defined by $G: L^2 (\mu) \rightarrow D$ via
\begin{equation}
\label{eq:galerkin_def}
G\nu = \sum_{k,j=1}^n \hat{S}^{-1}(k,j) \langle \chi_k, \nu \rangle_\mu \chi_j.
\end{equation}
\end{defi}

The matrix $\hat{S}$ is invertible since it is the Gramian matrix of linear independent functions.
In the easy case that the $\{\cfam\}$ are the characteristic functions belonging to a full partition $\{A_1,\dots,A_n\}$, equation \eqref{eq:galerkin_def} becomes \marginpar{``weighted'' orthogonal projection?}
\begin{equation*}
G\nu = \sum_{k=1}^n \frac{1}{\mu(A_k)} \langle \chi_k, \nu \rangle_\mu \chi_k,
\end{equation*}
since the $\chi_i$ are orthogonal which means that $\chi_k \chi_j = 1$ if $j=k$ and $0$ otherwise.

A Galerkin projection can be applied on the transfer operator of a Markov process.

\begin{defi}[Projected Transfer Operator]
Let $\Pcal := \Pcal^t$ be the transfer operator of a Markov process on a state space $\X$ with unique invariant measure $\mu$, $\{ \cfam \}$ be a partition of unity and $G$ the Galerkin projection onto the associated subspace $D$. Then an operator of the form
\begin{equation*}
G \Pcal G: L_\mu^2 (\X) \rightarrow D
\end{equation*}
is called \textit{projected transfer operator} and we abbreviate it by $G(\Pcal)$.
\end{defi}

\subsubsection*{Matrix Representation}

%\marginpar{def. GPG not $n \times n$, but $\infty \times n$?} only first definition
%As we are interested
Since we are interested in transitions inside of our smaller (projected) space, we want to propagate $n$-dimensional vectors by the projected transfer operator. For this reason, we consider now the projection of the \textit{restricted} transfer operator $G\Pcal_{\mid D} : D \rightarrow D$, which will be denoted $G(\Pcal)$ as well.
\marginpar{GPG same as $GP_D$?}
%It can be written as a $n \times n$-matrix in an interesting/useful way.

%remember/recall. described/represented
We remind that every linear map between finite-dimensional vector spaces can be represented by a matrix which is determined by chosen bases. \marginpar{see ..}
Thus we can write the projected transfer operator as a $n \times n$-matrix in the following useful way.
%a way that will be very useful later.

\begin{thm}
%sarich p.45
%nielsen p. 39
\label{thm:galerkin}
Let $\Pcal$ be the transfer operator of a Markov process, $\{ \cfam \}$ a partition of unity and $G(\Pcal)$ the Galerkin projection of the transfer operator onto the associated subspace.
%\marginpar{ansatz space vs subspace}
Then $G(\Pcal)$ has a matrix representation
%Galerkin projection onto $D$ has a matrix representation
%is defined by $Q: L^2 (\mu) \rightarrow D$ by
\begin{equation*}
P_c = T S\inv,
\end{equation*}
where
\begin{equation}
\label{eq:massmatrix}
S_{kj} =  \frac{\hat{S}(k,j)}{\langle \chi_k, \eins \rangle_\mu}
= \frac{\langle \chi_k, \chi_j \rangle_\mu}{\langle \chi_k, \eins \rangle_\mu}
\end{equation}
and \marginpar{$\eins_D \eins_\X$?}
\begin{equation*}
T_{kj} = \frac{\langle \chi_k, \Pcal \chi_j \rangle_\mu}{\langle \chi_k, \eins \rangle_\mu}.
\end{equation*}
%Both $S$ and $T$ are $n \times n$-matrices.
\end{thm}

\begin{proof}
%a/the matr. repr.?
Remember that $P_c$ is a (left) matrix representation of $G(\Pcal)$ with respect to a basis $\{\pfam\}$ of $D$ if for any function $f : D \rightarrow D$ with
\begin{equation*}
f = \sum_{i=1}^n \alpha_i\psi_i \ \textrm{ and } \ G(\Pcal) f = \sum_{i=1}^n \beta_i\psi_i
\end{equation*}
it holds that
\begin{equation}
\label{eq:representation1}
(\afam)P_c = (\bfam).
\end{equation}
We assume that \eqref{eq:representation1} holds and choose a basis $\{\pfam\}$ of $D$ with
\begin{equation*}
\psi_k = \frac{\chi_k}{\langle \chi_k, \eins_\X \rangle_\mu}.
\end{equation*}
By applying the projected transfer operator to $f$, we try to find the coefficient vector $(\bfam)$. We  exploit the definition of a Galerkin projection and the definition of our basis:
\begin{align*}
(G\Pcal G) f & =  \sum_{k=1}^n \alpha_k (G\Pcal) \psi_k \\
 & = \sum_{k,l,j=1}^n \alpha_k \hat{S}\inv (j,l) \langle \chi_j, \Pcal \psi_k \rangle_\mu \chi_l \\
 & = \sum_{k,l,j=1}^n \alpha_k \hat{S}\inv (j,l) \langle \chi_j, \Pcal \psi_k \rangle_\mu \langle \chi_l, 		      \eins_\X \rangle_\mu \psi_l \\
 & =  \sum_{l=1}^n \beta_l\psi_l.
\end{align*}
Therefore, we can compute the coefficients $\beta_l$ as
%of the image?
\begin{align*}
\beta_l & = \sum_{k,j=1}^n  \alpha_k \hat{S}^{-1} (j,l) \langle \chi_l, \eins_\X \rangle_\mu \langle \chi_j, \Pcal \psi_k \rangle_\mu  \\
 & = \sum_{k=1}^n \alpha_k \underbrace{\sum_{j=1}^n \hat{S}^{-1} (j,l) \langle \chi_l, \eins_\X \rangle_\mu \frac{\langle \chi_j, \Pcal \chi_k \rangle_\mu}{\langle \chi_k, \eins_\X \rangle_\mu}}_{P_{kl}}. \numberthis \label{eq:representation2}
\end{align*}
The underbraced term is equal to $P_{kl}$ because of \eqref{eq:representation1}. Now we compare it to the $(k,l)$-th entry of $TS\inv$ which is computed via matrix multiplication as
\begin{align*}
(TS^{-1})_{kl} & = \sum_{j=1}^n T_{kj} (S\inv)_{jl} \\
 &  = \sum_{j=1}^n \frac{\langle \chi_k, \Pcal \chi_j \rangle}{\langle \chi_k, \eins \rangle}
(\hat{S}\inv)_{jl} \langle \chi_j, \eins \rangle.
\end{align*}
We see that this corresponds to the $(k,l)$-th entry of $P_c$, compare with \eqref{eq:representation2}.
\end{proof}
%Nielsen p.40

\begin{thm}
The matrices $S$ and $T$ are stochastic. \marginpar{non-neg.}
\end{thm}
\begin{proof}
In order to be stochastic, each row must sum up to $1$. We exploit the partition of unity property $\sum_j \chi_j = 1$ for all $j$ and the aforementioned property $\Pcal  \eins =  \eins$ of a transfer operator:
\begin{equation*}
\sum_{j=1}^n S_{kj}
= \frac{\langle \chi_k, \sum_j\chi_j \rangle_\mu}{\langle \chi_k, \eins \rangle_\mu}
= \frac{\langle \chi_k, \mathbb{1} \rangle_\mu}{\langle \chi_k, \eins \rangle_\mu} =1,
\end{equation*}
\begin{equation*}
\sum_{j=1}^n T_{kj}
= \frac{\langle \chi_k, \Pcal \sum_j \chi_j \rangle_\mu}{\langle \chi_k, \eins \rangle_\mu}
= \frac{\langle \chi_k, \Pcal \mathbb{1} \rangle_\mu}{\langle \chi_k, \eins \rangle_\mu} = 1.
\end{equation*}
\end{proof}
Since $S$ and $T$ are both stochastic matrices, they have $\eins_D$ as right eigenvector to the eigenvalue $1$. It implies that the same holds for $P_c$, that is
\marginpar{can propagate prob. distr. (keeping norm)}
its rows sum up to $1$ and thus the product $TS\inv$ is \textbf{at least pseudostochastic}. But nonnegativity is not assured since inverting $S$ can provoke/cause negative entries. The non-negativity depends on the choice of the partition of unity. As we will see, there are examples such that $TS\inv$ is a stochastic matrix.
%provoke/evoke/produce/cause

\begin{thm}
\label{thm:lefteigenvector}
The matrix representation $P_c$ from theorem \ref{thm:galerkin} has the left eigenvector $\hat{\mu}$ with the entries
%left-eigenvector = invariant measure = stationary distribution/ measure of $P_c$
\begin{equation*}
\hat{\mu}_k = \langle \eins, \chi_k \rangle_\mu = \int_\X \chi_k(x) \mu(\diff x).
\end{equation*}
\end{thm}

\begin{proof}
We observe that $\hat{\mu}^T S = \hat{\mu}^T$ and $\hat{\mu}^T T = \hat{\mu}^T$ since
\begin{equation*}
(\hat{\mu}^T S)_j = \sum_{k=1}^n  \langle \eins, \chi_k \rangle_\mu \frac{\langle \chi_k, \chi_j \rangle_\mu}{\langle \chi_k, \eins \rangle_\mu}
= \langle \eins, \chi_j \rangle_\mu = \hat{\mu}_j
\end{equation*}
and
\begin{equation*}
(\hat{\mu}^T T)_j = \sum_{k=1}^n \langle \chi_k, \Pcal \chi_j \rangle_\mu = \langle \eins, \Pcal \chi_j \rangle_\mu
= ... = \hat{\mu}_j.
\end{equation*}
We can deduce that $\hat{\mu}^T P_c = \hat{\mu}^T TS^{-1} = \hat{\mu}^T S^{-1} = \hat{\mu}^T SS^{-1} = \hat{\mu}^T$.
\end{proof}

\begin{rem}
We remark that in theorem \ref{thm:galerkin} we stated the matrix representation of the \textit{(forward) propagator} of a projected Markov process. As the backward operator is the adjoint operator of the propagator,
the matrix representation of its projection is just the transpose $P_c^T = S^{-1}T$.
This result and its properties are computed completely analogously to the previous proofs.

$P_c$ propagates probability distributions, while $P_c^T$ propagates sets(?).
\end{rem}
%norm needs not be conserves for set propagation???
%left or right multiplication of vectors???

\subsubsection*{Example}

The easiest example we can consider is the special case of a full partition discretization, which would also be the first intuitive approach to cluster a state space: take subsets and define them to be one \textit{macro state}.
%intuitive first approach: to group subsets together

\begin{ex}[Full Partition Discretization]
Let $\{ \cfam \}$ be a partition of unity that is induced by a full partition, i.e. the $\chi_i$ are the characteristic functions of pairwise disjoint sets $A_i$ s.t. $\cup A_i = \X$. Then the matrix representation $P_c$ of $G\Pcal G$ is a stochastic matrix consisting of the transition probabilities between the partition sets, i.e.
\begin{equation*}
P_c(k,l) = \Prob(X_t \in A_l \mid X_0 \in X_k).
\end{equation*}
\end{ex}

\begin{proof}
The entries of the Gram matrix $\hat{S}$ are $\mu(A_k)$ on the diagonal and $0$ everywhere else.
We can deduce that $S$ is the unit matrix, while $P_c=T$ is a stochastic matrix. \marginpar{makes role of $S,T$ clear}
For the entries of $P_c$ we get:
\begin{equation*}
P_c(k,l)=
\frac{\langle \chi_l, \Pcal \chi_k \rangle_\mu}{\langle \chi_k, \eins_\X \rangle_\mu}
= \frac{\Prob_\mu (\{X_t \in A_l\} \cap \{X_0 \in A_k\})}{\Prob_\mu(X_0 \in A_k)}
= \Prob(X_t \in A_l \mid X_0 \in X_k).
\end{equation*}
\end{proof}

In this case, $P_c$ is a stochastic matrix and hence the transition matrix of a Markov chain. \marginpar{ref to 1-1-rel?}
Its state space consists just of the partition sets $A_i$. The stationary distribution of the so defined Markov chain $P_c$ is just the projection of the invariant measure $\mu$ onto $D$. \marginpar{what for?}

%This example makes clear that blablabla
For a full partition discretization, the matrix $S$ is a diagonal matrix. If we choose a partition of unity that is \textit{close} to a full partition, i.e. we choose \textit{almost characteristic functions}, then the matrix $S$ is not diagonal, but close to that. We will later see the consequences of that fact regarding to the examination/investigations of the \textit{rebinding effect}.
%more overlap: more rebinding

\subsubsection*{Properties of Galerkin Projection}

As the matrix representation of a projected transfer operator is in general \textbf{not} a stochastic matrix, %which is equivalent to being the transition matrix of a Markov chain (see ..),
we see immediately that the process can lose its Markovianity by projecting it (onto a subspace).
That is due to the one-to-one-relation between a stochastic matrix (transition matrix) and a Markov chain.
This possible loss of Markovianity is certainly a really undesirable effect. But before examining that later in section \ref{sec:recrossing}, let us now first analyze further, hopefully \textbf{nice}, properties of the matrix representation $P_c$.
\\

We already know that the matrices $S$ and $T$ from Theorem \ref{thm:galerkin} are stochastic matrices. This leads to some good properties of $P_c$:

\begin{itemize}
\item The eigenvalue $\lambda = 1$ of $P_c$ has the associated right-eigenvector $e = (1,\dots,1)^T$ and left-eigenvector $\hat{\mu}^T$ from theorem \ref{thm:lefteigenvector}.
\item If $\Pcal$ is self-adjoint in $L^2$, then $G\Pcal G$ as well. Then the matrices $S$ and $T$ are self-adjoint with respect to the discrete scalar product \marginpar{$P_c$ self-adj.?}
\marginpar{$\langle Av,w \rangle = \langle v, Aw \rangle$}
\begin{equation*}
\langle u, v \rangle_{\hat{\mu}} = \sum_{i=1}^n u_i v_i \hat{\mu}_i.
\end{equation*}
Since self-adjointness of the operator is equivalent to reversibility of the corresponding process, see theorem \ref{thm:selfadjoint_reversible}, detailed balance equation (e.g. $\hat{\mu}_k T_{kl} = \hat{\mu}_l T_{lk}$ for all $k,l =1,\dots, n$) is fulfilled for both $S$ and $T$.
\marginpar{thus, eigenv. of S and T real and in $[-1,1]$}
\item If the transfer operator has a simple and dominant eigenvalue $1$ and the continuous part of the spectrum is bounded away from the discrete part, then the process is irreducible and aperiodic which is inherited by the matrix $T$. In particular $T$ has the simple and dominant eigenvalue $\lambda=1$ which is the only eigenvalue with $|\lambda|=1$ and the discrete invariant density $\hat{\mu}$ is the unique invariant density of $T$.
\item As seen in the last example a full-partition projection yields the transition matrix $P_c=T$ of a Markov chain describing transitions between the partition sets.
\end{itemize}

\subsubsection*{Summary}
%p.65
Summarizing, the discretization of the propagator can be interpreted as a \textit{coarse graining} procedure.
%in the sense that ...
... blabla ......
The discretization maintains/keeps/inherits the most important properties of the transfer operator/propagator.

\subsubsection*{Projected infinitesimal generator}

The Galerkin projection of an infinitesimal generator yields a similar matrix representation as it has been the case for the transfer operator. It can also be written as the product of two stochastic matrices, one of them inverted.

\begin{thm}
Let $\Qcal: L^2(\mu) \rightarrow L^2(\mu)$ be a generator of a semigroup of transfer operators with unique invariant measure $\mu$ and satisfying $\Qcal\eins_\X = 0$.
Let $\chi$ be a partition of unity with a projection $G$ onto the associated subspace spanned by $\chi$.
Then the projected generator $G \Qcal G$ has the matrix representation $Q = RS^{-1}$ with the stochastic mass matrix $S$ from \eqref{eq:massmatrix} and \marginpar{not commutative}
\begin{equation*}
R(k,j) = \frac{\langle \chi_k, \Qcal\chi_j \rangle_\mu}{\langle \chi_k, \eins \rangle_\mu}
\end{equation*}
The eigenvalue problem of $Q$ is equivalent to the generalized eigenvalue problem $Ru = \Lambda Su$.
For both $Q$ and $R$ the largest eigenvalue is $\lambda = 0$. The associated right eigenvector is $e=(1,\dots,1)^T$, the associated left eigenvector is $\hat{\mu}^T$ from theorem \ref{thm:lefteigenvector}.
\end{thm}
\begin{proof}
The matrix representation of $G\Qcal G$ can be shown similar to the proof of theorem \ref{thm:galerkin}. For the other properties see...
\end{proof}

There are obiously many possibilities/options to make a Galerkin discretization/projection of the propagator/generator of a given process because we can define it on an arbitrary partition of unity $\cfam$.
%transfer operator respectively infinitesimal generator 
%\marginpar{propagator, generator}, desirable/favorable
In chapter \ref{chap:meta} we are going to see which choice of $\chi$ gives us a \textit{good} discretization of our process in the sense that it maintains certain desired properties; in our case the long-time behaviour of the process using so called \textit{metastability}.
%n can be determined by regarding the spectrum of the transfer operator
%In chapter \ref{chap:meta} we are going to show how the number of clusters (= metastable sets) can be %determined using the spectrum of the transfer operator. Furthermore we will show how to construct %functions $\cfam$ (=membership functions) that give us a decomposition which is as good as possible.
%(linear combination of eigenfunctions).