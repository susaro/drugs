\begin{figure}[h]
%	\label{fig:reb_example_nonrev}
	\centering
	\subfigure[The minimal rebinding effect compared to the degree of non-reversibility of the clustered system $Q_c$.]{\includegraphics[width=0.31\textwidth]{figures/rebinding/200_random_schur/reb_opt_nonrev_200_random.eps}}
	\hspace{1pt}
	\subfigure[The minimal and the real rebinding effect compared to the degree of non-reversibility of the clustered system $Q_c$.]{\includegraphics[width=0.31\textwidth]{figures/rebinding/200_random_schur/reb_nonrev_200_random.eps}}
	%, increasing the probability of a fast \textbf{rebinding}.
	\hspace{1pt}
	\subfigure[The minimal rebinding effect $\det(\Sopt)$ compared to the real rebinding effect $\det(\Sreal)$ included in $Q_c$.]{\includegraphics[width=0.31\textwidth]{figures/rebinding/200_random_schur/reb_200_random.eps}}
	%\caption{The system described by the transition matrix $P$ is clustered with $200$ randomly generated feasible transformation matrices $A$ for the values $\epsilon = 0$, $\delta = 0$.} % in order to compare some important parameters
	\hspace{20pt}

	\subfigure[The minimal rebinding effect compared to the degree of non-reversibility of the clustered system $Q_c$.]{\includegraphics[width=0.31\textwidth]{figures/rebinding/200_random_schur/epsilon0.004/reb_opt_nonrev_200_random.eps}}
	\hspace{1pt}
	\subfigure[The minimal and the real rebinding effect compared to the degree of non-reversibility of the clustered system $Q_c$.]{\includegraphics[width=0.31\textwidth]{figures/rebinding/200_random_schur/epsilon0.004/reb_nonrev_200_random.eps}}
	%, increasing the probability of a fast \textbf{rebinding}.
	\hspace{1pt}
	\subfigure[The minimal rebinding effect $\det(\Sopt)$ compared to the real rebinding effect $\det(\Sreal)$ included in $Q_c$.]{\includegraphics[width=0.31\textwidth]{figures/rebinding/200_random_schur/epsilon0.004/reb_200_random.eps}}
	%\caption{The system described by the transition matrix $P$ is clustered with $200$ randomly generated feasible transformation matrices $A$ for the values $\epsilon = 0.004$, $\delta = 0$.} % in order to compare some important parameters
	\hspace{20pt}
	
	\subfigure[The minimal rebinding effect compared to the degree of non-reversibility of the clustered system $Q_c$.]{\includegraphics[width=0.31\textwidth]{figures/rebinding/200_random_schur/epsilon0.004delta0.01/reb_opt_nonrev_200_random.eps}}
	\hspace{1pt}
	\subfigure[The minimal and the real rebinding effect compared to the degree of non-reversibility of the clustered system $Q_c$.]{\includegraphics[width=0.31\textwidth]{figures/rebinding/200_random_schur/epsilon0.004delta0.01/reb_nonrev_200_random.eps}}
	%, increasing the probability of a fast \textbf{rebinding}.
	\hspace{1pt}
	\subfigure[The minimal rebinding effect $\det(\Sopt)$ compared to the real rebinding effect $\det(\Sreal)$ included in $Q_c$.]{\includegraphics[width=0.31\textwidth]{figures/rebinding/200_random_schur/epsilon0.004delta0.01/reb_200_random.eps}}
	%\caption{The system described by the transition matrix $P$ is clustered with $200$ randomly generated feasible transformation matrices $A$ for the values $\epsilon = 0.004$, $\delta = 0.01$.} % in order to compare some important parameters
	\hspace{20pt}
	
	\subfigure[The minimal rebinding effect compared to the degree of non-reversibility of the clustered system $Q_c$.]{\includegraphics[width=0.31\textwidth]{figures/rebinding/200_random_schur/epsilon0.002gamma0.01/reb_opt_nonrev_200_random.eps}}
	\hspace{1pt}
	\subfigure[The minimal and the real rebinding effect compared to the degree of non-reversibility of the clustered system $Q_c$.]{\includegraphics[width=0.31\textwidth]{figures/rebinding/200_random_schur/epsilon0.002gamma0.01/reb_nonrev_200_random.eps}}
	%, increasing the probability of a fast \textbf{rebinding}.
	\hspace{1pt}
	\subfigure[The minimal rebinding effect $\det(\Sopt)$ compared to the real rebinding effect $\det(\Sreal)$ included in $Q_c$.]{\includegraphics[width=0.31\textwidth]{figures/rebinding/200_random_schur/epsilon0.002gamma0.01/reb_200_random.eps}}
	%\caption{The system described by the transition matrix $P$ is clustered with $200$ randomly generated feasible transformation matrices $A$ for the values $\epsilon = 0.002$, $\gamma = 0.01$.} % in order to compare some important parameters
\end{figure}