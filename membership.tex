\section{Fuzzy Clustering via Membership Functions}
%Soft/ Fuzzy
%\section{Decomposition via membership functions}
%\section{Metastable decomposition via membership functions}
%\section{Set-based vs Function-based Approach}

The above considerations (metastable decomposition induced by zeros of eigenfunctions) result in a metastable full decomposition of the state space, i.e. each state is assigned to exactly one of the partition sets.
We will see now, that there exist better solutions, considering/ including the fact that transition regions can belong to several metastable sets. So for this new approach (even though already examined/investigated/analyzed in recent research, see  ...), there may be some overlap in the assignment of states to metastable sets.

% a transition region can be %assigned to several macro states with different weight(?).

\subsubsection*{Set-based vs. Function-based Approach}
An easy intuitive approach to decompose the state space would be to determine a certain number of metastable sets which form a full partition of the state space, such that each state is assigned to exactly one of the metastable sets.
%each partition corresponds to one conformation/ metastable set.
The problem with that approach is that also the transition regions of the process would have to be assigned to one of these partition sets. But why would you assign a state in a transition region to one adjacent metastable set and not to the/one other? So such an assignment is not a rigoruous description of actual behaviour of the process.
%does not make so much sense.

So this \textit{set-based approach} of decomposing the process has been replaced by the \textit{function-based approach}.
That means that each state of the process is assigned with a certain probability to each metastable set.
That makes sense, because a state in a transition region can go with certain probabilities to different conformations.

\subsubsection*{Membership functions (=almost characteristic function)}

%reversible
Assuming we have already determined that our process consists of $n$ metastable sets (by knowing its $n$ dominant eigenvalues).
%the number of metastable sets by 
%we are considering a stochastic process with $n$ dominant eigenvalues
We will follow the approach of \cite{weber2006meshless} by defining macro states as \textit{overlapping partial densities}. They can be identified by membership functions $\cfam : \X \rightarrow [0,1]$. Each state of the original state space shall be assigned to the different macro states with a certain \textit{degree of membership}.
%using membership functions (linear combinations of eigenfunctions). no??

\begin{defi}
membership function $\chi_i$ = partition of unity (see 2.4) and nonnegative
\end{defi}

Membership functions can be used to decompose the space into metastable sets. In this case, the assignment of a state to a metastable set must not be unique, but a state can belong to different metastable sets with certain probabilities(?). That model takes into consideration
%takes into account
the existence of transitions regions which cannot be uniquely assigned to one macro state/ metastable set.
%detect a 

%membership functions: continuous and overlapping

Since membership functions form a partition of unity, we can apply the Galerkin projection as defined in section \ref{sec:galerkin}. That gives us a Markov State Model where each (macro) state is a metastable set of the original process. \marginpar{so far: membership fct indep. of metastable set}
%As membership functions are a partition of unity, we can 1) create a metastable decomposition (maybe with %overlap, so no partition) and 2) apply Galerkin Projection on it (partition of unity necessary).

\begin{ex}
A characteristic function is a membership function. It induces a full partition as metastable decomposition.
\end{ex}
Individual eigenfunctions $\Xcal$ do not overlap since they are orthogonal. But the membership functions $\chi_i$ as linear combinations of the dominant eigenfunctions might have an overlap.

\subsubsection*{Statistical Weights}

For each macrostate we can define a statistical weight
\begin{equation*}
w_i = \int_\Omega \chi_i(q) \pi_q(q) \diff q,
\end{equation*}
i.e. ....

$D = \mathrm{diag}(w_1,\dots,w_n)$ is the diagonal matrix of the statistical weights of the membership functions

\subsubsection*{Perron Cluster Analysis}

The  whole approach  is  called Perron  cluster  analysis, its agorithmic  realization is  abbreviated  as  PCCA  (from  Perron  Custer  Custer  Analysis). \marginpar{PCCA+: robust}
\\

So now we need to choose/determine convenient/suitable membership functions s.t. the metastable decomposition becomes as good as possible. Normed eigenfunctions are one possibility (no overlap). But better: linear combinations of eigenfunctions. \marginpar{only for finite matrices? or also cont. operator w/ eigenfct.?}


Perron Cluster Analysis has been developed by Deuflhard et al\cite{deuflhard2000identification} which used the sign structure of the dominant eigenvalues of the stochastic matrix. \marginpar{set-based approach?} That approach has been approved by
Deuflhard and Weber\cite{deuflhard2005robust} \marginpar{oder Weber und Galliat?}
who transformed the system of eigenvectors into a system of membership functions which results in a soft/fuzzy clustering of the state space of the original process; their algorithm is called PCCA+ (Perron Cluster Cluster Analysis).

The required membership functions can be built as a linear combination $A\Xcal$ of the dominant eigenfunctions (convex problem?).
The algorithm PCCA+ computes the matrix $A$ which results in the best metastable decomposition.
%yields
\\

We are considering the set of dominant eigenvalues $\{\lfam\}$ with the correponding set of eigenfunctions $\Xcal = \{\xfam\}$. They fulfill the eigenvalue problem $\Pcal(\tau) \Xcal = \Xcal \Lambda$ of the transfer operator $\Pcal(\tau)$, where $\Lambda = \mathrm{diag}(\lfam)$.
\\

Let $\Xcal$ be the eigenvector matrix, \marginpar{why not eigenfct.?} i.e. the $i$-th column of $\Xcal$ is an eigenvector corresponding to the eigenvalue $\lambda_i$. \marginpar{only dom. spec.}
Weber showed in \cite{weber2011subspace} that the choice $\chi = \Xcal A$ (linear combination of the eigenfunctions $\Xcal$) preserves the Markov property, i.e. \marginpar{jede konvexe lin.komb.?} \marginpar{bzw even: discretization error vanishes}
%Markovianity
\begin{equation*}
\chi_j(q) = \sum_{i=1}^n A_{ij} \Xcal_i(q), \ j=1,\dots,n,
\end{equation*}
where $A = \{A_{ij}\}_{i,j=1,\dots,n} \in \R^{n\times n}$ is the solution of a convex maximization problem (PCCA+), see Weber\cite{weber2006meshless}. In order to fulfill the partition of unity of each $\chi_i$, the matrix $A$ has to be choosen s.t. $\chi$ is row-stochastic.

\subsubsection*{Optimal decomposition (PCCA+)}

Since there are so many possibilities membership functions (all convex combinations of dominant eigenfunctions), we want to find the solution which gives us the best decomposition into metastable sets; i.e. ...

%Second largest eigenvalue results in the best decomposition and so on ...

%But Better: linear combination of eigenfunctions (might have an overlap) (membership functions, PCCA+).
%That will be described more in the next subsection 2.4 where, in aim to get a really good MSM, we try to %find the best possible decomposition (?) into metastable sets (? not a partition anymore) 

\subsubsection*{PCCA+}
Finding membership functions which result in optimal solution for metastable decomposition (but PCCA+ works only on finite state spaces(?).
\\
%PCCA+ can be used to perform the computation of the best metastable decomp. as described above

%Noe Lecture 4 (MSM 2015)
There are infinitely many transformations A of the eigenvectors resulting in a soft membership matrix
$\chi$ satisfying the positivity and partition of unity constraints.  Consequently, we have to determine the transformation A that satisfies some optimality condition.

Do we need PCCA+ to find the best(?) membership functions. And with the help of these membership functions we can apply a Galerkin Projection?