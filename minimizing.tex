\section{Minimizing the Rebinding Effect}
\label{sec:minimizing}

So far, we know that the matrix $S$ from \eqref{eq:projection_presentation} gives a measure for the quantity of the rebinding effect, i.e. being close to the identity matrix implies a low rebinding, while high outer diagonal elements of $S$ (overlap?) imply a high rebinding effect.
But we don't know yet the meaning of this effect for the whole process.
%So before starting with any kind of optimization problems,
So we will at first set the rebinding effect (respectively the matrix $S$) in relation to the stability of a system/process.
%\marginpar{2 adv.: stability and efficiency of drugs}
Afterwards, we will formulate an optimization problem in order to get a lower bound for the rebinding effect.

%Is rebinding effect good/bad/desirable?
%As mentioned before, the rebinding effect is a desired/favorable property for some processes, e.g. in drug %design, where high rebinding increases the efficiency of a drug.
%In order to get high rebinding effects, we will now derive a lower bound for it; i.e. minimize the rebinding %effect.

As the computation of eigenfunctions of a continuous operator $\Qcal$ is an extensive task, we will assume in the further course that the transition rates can be measured experimentally.
%of a process
Thus, we will examine a given transition rate matrix $Q_c$.
%a process belonging/corresponding to a given transition rate matrix?

%For reversible processes, this problem is solved by Weber and Fackeldey \cite{weber2014} with the
%spectral approach.

\subsubsection*{Stability of the system/process in terms of determinants}

If the eigenvalues $\xi_i$ of $Q_c$ are close to $0$, then the macro states are very stable in the sense that the probability to stay inside of such a state is close to $1$.
The trace of $Q_c$ corresponds to the sum of the dominant eigenvalues of $\Qcal$.
%Remember that the largest eigenvalue ist $0$; the other eigenvalues arbitrarily negative numbers
Thus, we can measure the \textit{stability} of the molecular system by considering $F := - \trace(Q_c)$.
If $F$ is high, then the process is fast and less stable. If $F$ is close to $0$, then the process is slow and very stable. \marginpar{trace indep. of $A$?}
We want to set the indicator for stability $F$ in relation to the matrices $S$ and $T$ from theorem \ref{thm:galerkin} (matrix representation of Galerkin projection).
\begin{lem}
If $Q_c$ is a projected infinitesimal generator of a process and $P_c(\tau)$ the corresponding projected transfer operator with a matrix representation $P_c(\tau) = S^{-1}T$ from theorem \ref{thm:galerkin}, then the quantity $F := - \trace(Q_c)$ can be measured by
%the stability $F$?
\begin{equation*}
\label{thm:stability}
F = \tau^{-1}(\log(\det(S)) - \log(\det(T))).
\end{equation*}
%if $\tau$ is the time-step of the corresponding (discretized) transition matrix $P(\tau)$.
\end{lem}
\begin{proof}
We use the relation $\exp(\trace()) = \det(\exp())$, the fact that $Q_c$ ``generates'' $P_c(\tau)$, theorem \ref{thm:galerkin} and multiplicativity of determinants to see that \marginpar{see ..}
\marginpar{why $\exp(\tau Q_c) = P_c(\tau)$?}
\begin{align*}
F & = - \trace(Q_c) \\
  & = -  \tau^{-1}\log(\exp(\trace(\tau Q_c))) \\
  & = -  \tau^{-1}\log(\det(\exp(\tau Q_c)))  \\
  & = -  \tau^{-1} \log(\det(P_c(\tau))) \\
  & = \tau^{-1}(\log(\det(S)) - \log(\det(T))).
\end{align*}
\end{proof}

%Assume det S/T in [0,1]? I.e. diagonal elements more influence than other entries -> log only def. on pos.
Thus, both determinants of the stochastic matrices $S$ and $T$  influence the stability of the system, but in converse directions.
%We remark that both determinants lie in $[-1,1]$ since both matrices $S$ and $T$ are stochastic matrices.

If $\det(T)$ is close to $1$, then $F$ is low and thus the process is stable/slow.
\marginpar{assuming $\det$ between $0$ and $1$?}
%($F = \log\det(S) + 0$ is low).
%That makes sense since we wanted the process to be clustered into metastable sets, so $T$ having large %eigenvalues on its diagonal, which results in a high determinant.
If $\det(T)$ is close to $0$, then the process is unstable/fast, since $F$ is high.
%$F = \log\det(S) + \infty$ is high).
%Remember that $T$ was interpreted to be the stochastic matrix describing the propagation(?) of the %process, while $S$ was the matrix ``disturbing'' that.
That makes sense because a high determinant of $T$ can be interpreted as ``good'' metastability and thus corresponds to a slower/stable process, while a low determinant of $T$ (``bad'' metastability) makes the process faster/unstable.
%Thus, it makes sense that a high determinant of $T$ (``good'' metastability) corresponds to a slower/stable
%process, while a low determinant of $T$ (``bad'' metastability) makes the process faster/unstable.
%metastability measured by trace? not det?

On the other hand, if $\det(S)$ is close to $1$ (almost unit matrix), then $F$
%$F = 0 + \log\det T$
is high, i.e. the process is unstable/fast.
If $\det(S)$ is close to $0$ (much overlap), then $F$ is low, i.e. the process is stable/slow.
%$F = - \infty + \log\det T$
That means that a higher overlap in $S$ leads to a slower process!
This relation is not as obvious at first sight.

As we figured out in section \ref{sec:projection}, the rebinding effect can be measured by the matrix $S$. If that matrix is close to the identity matrix, i.e. having a determinant close to $1$, yields a low rebinding. Large deviation from the identity matrix, i.e. having higher outer diagonal elements, i.e. having a smaller determinant, results in high rebinding.

%We can deduce now, that a system with strong rebinding, i.e. $S$ deviating from the unity matrix/ having %small determinant (much overlap), is more stable.

Combining these two properties of $S$, we can deduce that a high rebinding effect corresponds to a high stability of the system. This can be achieved by high outer diagonal elements/deviation from identity, i.e. by a high overlap of the membership functions(?).
%At first sight, this is an astonishing result as...
%We can deduce that..
%A system with strong rebinding is more stable!

%As we wanted $P_c(\tau)$ to be clustered by metastable sets, we can assume its eigenvalues to be not too %far away from $1$, let's say at least positive, which should be the case as well for $S$ and $T$.
%\marginpar{what about dominant cycles (NESS)? then $\det = -1$ could be interesting?}
%We also assume T to be more "dominant" in the sense that T should determine the actual behaviour of the
%process. Whereas S is only an "asjusting" factor which adds some "disturbance".
%Thus det T > det S??

%In a nutshell/in summary, there are two factors(?) which lead to an increased/high stability of a system.
%First, a process is stable if it is clustered into metastable sets, which results in a clustered matrix with large %eigenvalues on its diagonal and small outer diagonal elements; thus a large determinant of $T$.
%Secondly, a large overlap of the membership functions(?)/high rebinding effect, corresponding to high outer %diagonal elements of $S$, result in a stable process; thus a small determinant of $S$.

%Conclusion: A process is stable if it is clustered into metastable sets, which results in a clustered matrix with %large eigenvalues on its diagonal (high determinant of $T$). Another factor which makes the process stable %is a large overlap of the membership functions(?), contained in the matrix $S$, having a low determinant %(much overlap).

\subsubsection*{Finding a lower bound for the rebinding effect}

What is the meaning of the previous results/relation of $S$ to stability of the system?
%If we have a determinant $\det(S) = 1$, we have no rebinding.
On the one hand, the rebinding effect increases bindings(?). \marginpar{bad}
On the other hand, it increases the stability of a system. Thus, we we are interested in a high/increased rebinding effect.
%it would be nice to get a small determinant of $S$.
We are going to compute a lower bound for it, in order to know
how much rebinding there will be \textit{at least}.
%guaranteed
%The smaller this determinant, the higher the rebinding effect. As we are interested in a high/increased %rebinding effect, it would be nice to get a small determinant of $S$.

In order to do so, let us first of all remember how $S$ is determined. We were given a transfer operator $\Pcal$ which was projected onto a finite-dimensional state space via membership functions $\chi_i$. These membership functions have been computed as a linear combination of the eigenfunctions with a regular matrix $A$. Thus, the choice of the matrix $A$ determines $S$ respectively the size of its determinant.

So far, $A$ was assumed to be computed via PCCA+, i.e. such that the result is an optimal metastable decomposition/clustering.
Now, we want to take into consideration the set of all possible, \textit{feasible}, matrices $A$, to see if different choices result in a better/higher or worse/less rebinding.
%We want to know if there are possible choices of $A$ resulting in a ``better'', i.e. higher, rebinding effect.

%Task:
%Thus, we want to formulate an optimization problem to find out which choice of $A$ results in the ``best'', %i.e. highest, rebinding effect, measured by an \textit{optimal matrix} $\Sopt$.
Thus, we are going to formulate an optimization problem to find out which choice of $A$ results in the lowest
\marginpar{``worst''}
rebinding effect, measured by an \textit{optimal matrix} $\Sopt$, in order to know how much rebinding we are \textit{guaranteed} \textbf{at least} for a given process.
This problem is equivalent to finding the largest possible determinant of $S$.

\subsubsection*{Optimization Problem (Maximizing determinant of $S$)}

Since $Q_c$ has the same eigenvalues as $\Qcal$, the eigenvalue problem of $Q_c$ is given by
\begin{equation*}
Q_c X = X \Xi,
\end{equation*}
where the first column of $X$ corresponds to the first eigenvector $X_1 := (1,\dots, 1)^T$.
By \eqref{eq:galerkin_infinitesimal}, we see that $A^{-1}$ is an eigenvector matrix of $Q_c$ as well.
Therefore, the columns of the matrix $A^{-1}$ consists of multiples of the eigenvectors $X_i$. So we have
\begin{equation*}
A^{-1} =
\begin{pmatrix}
1 	  & & & \\
\vdots & \alpha_2 X_2 & \cdots & \alpha_3 X_3 \\
1	  & & &
\end{pmatrix}
\end{equation*}
with $\alpha_2,\dots,\alpha_n \in \R$. We know from theorem \ref{thm:stability} that a $\det(S)$ close to $1$ results in a low rebinding effect. Thus, in order to find a lower bound for the rebinding effect, we try to maximize $\det(S)$, or equivalently minimize $|\det(S)-1|$, since $S$ is a stochastic matrix having $1$ as largest possible determinant.
The \textit{objective function} of our optimization problem is then given by
\begin{equation}
\label{eq:optimization}
\mbox{
\boxed{ \min_{\alpha_1, \dots, \alpha_n \in \R} |\det(S) -1|},
}
\end{equation}
where we have to include several \textit{side constraints}. As the inverse matrix $A^{-1}$ consists of linear combinations of eigenvectors $X_i$, we have to consider
\begin{equation*}
\mbox{
\boxed{ \alpha_1 = 1 \ \ \ \mathrm{and} \ \ \ A_{ij}^{-1} = \alpha_i X_{ij} \ \ \forall i,j}.
}
\end{equation*}
Furthermore, $S$ is a stochastic matrix, see theorem \ref{thm:galerkin}, and its structure is given in terms of the linear transformation matrix $A$, so we have two further constraints
\marginpar{row-sum $1$ included in formula?}
\begin{equation*}
\mbox{
\boxed{S = D^{-1} A^TA \ \ \ \mathrm{and} \ \ \ S_{ij} \geq 0 \ \ \forall i,j}.
}
\end{equation*}
A \textit{feasible solution} of this optimization problem is a matrix $S$ fullfilling all side contraints, but not necessarily being an optimum.
% For instance, that could be a solution computed via PCCA+.?? fullfills really all constraints???

\subsubsection*{Interpretation}
\marginpar{explain overlap}
In the last section, we mentioned how the matrix $S$ describes the \textit{overlap} of the membership functions. \marginpar{???}
For this reason, any feasible solution of the optimization problem \eqref{eq:optimization} will be called a \textit{real overlap matrix} $\Sreal$, while an actual optimum will be called an \textit{optimal overlap matrix} $\Sopt$. Clearly, we get $\det(\Sreal) \leq \det(\Sopt) \leq 1$.

The real ocurring rebinding effect is high if the determinant of $\Sreal$ is low. Thus, a small determinant of $\Sopt$ increases the rebinding effect, while a large determinant of $\Sopt$ gives us only few information about the quantity(?) of the rebinding effect, it could be large or small.

Unfortunately, for a reversible $Q_c$, the solution of optimization problem \eqref{eq:optimization} gives us no information, as the following theorem shows.
\begin{thm}[Weber and Fackeldey{\cite[Theorem 1]{weber2014}}]
Let $Q_c \in \R^{n \times n}$ be a reversible matrix that stems from a clustering with positive definite overlap matrix $S$. Then there exists a matrix $A \in \R^{n \times n}$ in optimization problem \eqref{eq:optimization} such that $\det(\Sopt) = 1$.
%\det(D^{-1}A^TA)=1$.
\end{thm}
\begin{proof}
It is enough to show that for a given reversible $Q_c$, we can find a matrix $A$ fulfilling all constraints such that $S=D\inv A^TA$ is equal to the identity matrix $I$.
\\

Assume that there is a regular matrix $B$, such that $Q_c = B\inv \Xi B$.

Since $Q_c$ is reversible, we have $DQ_c = Q_c^T D$, see section \ref{sec:galerkin}, and thus
\begin{equation*}
DB\inv\Xi B = B^T\Xi^T B^{-T}D.
\end{equation*}
Now let $C:= B^{-T}D$, then we get
\begin{equation*}
Q_c = \dots = C\inv \Xi C.
\end{equation*}
$\dots$

Have a real positive matrix $M = \diag (m_1,\dots,m_n)$ \marginpar{???} and therefore a real positive diagonal matrix $\widetilde{M} = \diag (\sqrt{m_1}, \dots, \sqrt{m_n})$.

$\dots$

Let $A:= \widetilde{M}\inv B$. Show: $A$ fulfills the constraints of \eqref{eq:optimization}, in order that $S$ is a feasible matrix. Then
\begin{align*}
S = D\inv A^TA  & = D\inv B^T \widetilde{M}\inv \widetilde{M}\inv B \\
			 & = D\inv B^T M\inv B \\
		 	& = C\inv M\inv B         \\
			& = B\inv MM\inv B = I.
\end{align*}
Since all contraints of \eqref{eq:optimization} are fulfilled, $S$ is a feasible matrix with $\det(S) =1$.
%Determinants of stochastic matrices are inside $[-1,1]$. Hence maximality of
\end{proof}

This theorem does \textbf{not} mean that a reversible process has no rebinding effect. It just means that for \textbf{every} reversible process, it is possible to find a clustering with no rebinding.

For instance, if we computed a clustering of a reversible process via PCCA+, then it could be the case that we have a low determinant of $S$, i.e. a high rebinding. But the optimization problem \eqref{eq:optimization} gave us the lower bound of no rebinding, so it gave us no information about the rebinding of a particular/concrete clustering.
%it was just a superfluous/useless bound. \marginpar{no information}

\subsubsection*{Linear Optimization Problem (Maximizing trace of $S$)}
Now we present a different formulation of the above optimization problem \eqref{eq:optimization}. We will slightly change the objective function and turn the problem into a \textit{linear optimization problem}.
As a special case of the class of convex optimization problems, they have the nice property that any local optimum is also a global optimum. \marginpar{and easier to solve?}

We want to \textit{minimize} the rebinding effect, i.e. we want to give a bound for how large the rebinding effect is \textit{at least} (lower bound for rebinding effect). The closer the matrix $S$ is to the identity matrix, the smaller is the rebinding effect.
A matrix is close to the identity matrix, if its determinant is close to $1$ or (equivalently) if its trace is close to $n$.
So in order to compute the minimal rebinding effect, we can either maximize the determinant (get it as close as possible to $1$) or maximize the trace of $S$ (get it as close as possible to $n$), as the following theorem shows.
\begin{thm}
In optimization problem \eqref{eq:optimization}, we have $\det(S) \leq 1$ and $\trace(S) \leq n$ with equality if and only if $S$ is the identity/unit matrix.
\end{thm}
\begin{proof}
\end{proof}

So instead of maximizing $\det(S)$, we can also maximize $\trace(S)$. In order to do so, let us first make some further observations about the eigenvectors of $Q_c$.
We already found out that $A^{-1}$ is a right eigenvector matrix of $Q_c$, with vectors being linear combinations of the eigenvectors $X_i$.
Similarly, the matrix $A$ is a \textit{left} eigenvector matrix of $Q_c$, with row vectors being linear combinations of the eigenvectors $Y_i$. That fact can be expressed as
\begin{equation*}
A= \tilde{U}Y^T =
\begin{pmatrix}
 & & \tilde{\alpha_1}Y_1   & &  \\
 & & \vdots                       & &   \\
 & & \tilde{\alpha_n}Y_n   & &
\end{pmatrix},
\end{equation*}
where each $Y_i$ is a left eigenvector of $Q_c$ (row vector) and the $\tilde{\alpha}_i \in \R$ are again some optimization parameters.
The first eigenvector $Y_i$ corresponds to the leading eigenvalue $\xi_1 = 0$ and is thus the stationary density of the process. The first row of $A$ consists of the statistical weights of the clusters \marginpar{?}
and therefore we have again $\tilde{\alpha}_i = 1$.
With these notations we can write the new objective function as
\begin{align*}
\trace(S) & = \trace(D^{-1} A^T A) \\
              & = \trace(D^{-1}Y \tilde{U}^2 Y^T) \\
              & = \sum_{i=1}^n \sum_{k=1}^n \tilde{\alpha}_k^2 \frac{y_{ik}^2}{y_{k1}}.
\end{align*}
The side constraints remain the same, i.e. $S_{ij} = \dots \geq 0$. \marginpar{why $i \neq j$?}
Let $\beta = (\beta_1,\dots, \beta_n)$ with $\beta_i = \tilde{\alpha}_i^2$.
Then the linear optimization problem of maximizing $\trace(S)$ is given by

\begin{equation}
\label{eq:optimization_linear}
\mbox{
	\boxed{ \max_{\beta} \sum_{k=1}^n \beta_k \left( \sum_{i=1}^n 			\frac{y_{ik}^2}{y_{k1}} \right)},
}
\end{equation}
fullfilling the side contraints
\begin{equation*}
\mbox{
	\boxed{ \beta_i \geq 0, \ \beta_1 = 1}
}
\end{equation*}
and
\begin{equation*}
\mbox{
	\boxed{ \sum_{k=1}^n \beta_k y_{ik} y_{jk} \geq 0}.
}
\end{equation*}

At first sight, this second formulation of the optimization problem might seem a bit more complex/confusing since we introduced several new matrices and variables.
But in fact, the only change is that we maximize now the trace instead of the determinant (trace is easier to compute as it is just a sum).
This formulation is better since, we have a \textit{linear} program, which makes it easier to solve. And we have fewer contraints than before, because we merged some of the contraints into the objective function.

Let $B = \mathrm{diag}(\beta_1,\dots, \beta_n)$. Then a solution $\beta$ of \eqref{eq:optimization_linear} gives an optimal matrix $\Sopt = D^{-1} YBY^T$ resulting in the smallest possible rebinding effect.

%We know that the magnitude of the determinant of a stochastic matrix must be between 0 and 1 inclusive. %It's equal to 1 if and only if the matrix is a permutation matrix, with the determinant itself being equal to 1 %for an even permutation, and -1 for an odd permutation.

%So the closer the determinant of a stochastic matrix is to 1, the slower the transitions reach the steady state. %The closer the determinant is to 0, the faster the transitions reach steady state. 

\subsubsection*{Conclusion}

A nontrivial rebinding effect can be estimated only if the kinetics $Q_c$ of a system is nonreversible. \marginpar{why?}