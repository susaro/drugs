\chapter*{Outlook} %aktueller Stand der Forschung
%As the main topics of this thesis are current research topics, we give a short outlook of the current projects etc...

This thesis combines two research topics that are highly discussed recently.
%Firstly, the general field of molecular design with its particular applications like drug design and the rebinding effect...
The general field of molecular design, in particular applications in drug design like the rebinding effect, as well as the analysis of nonreversible processes are ongoing research topics with 


\subparagraph*{Molecular Design}

Recently: reduction of dimension using metastable decompositions (PCCA+) helped to perform simulations which would have been impossible on the larger (cont.) state space.

Outlook: But still it is only possible to simulate on ...timescales.. longer timescales like several seconds, like needed for protein folding etc, are still infeasible, but with increasing computing power of supercomputers become more and more realistic.

(Noe Weber..)

\subparagraph*{Rebinding Effect}

In order to design drug molecules, it is important to know the exact binding affinity of a given system (set of ligands). Therefore the knowledge of the occuring rebinding effect can help to improve this design process/..

\subparagraph*{Multivalence}

The kinetics and design of multivalent processes is a current research project of the ``Computational Molecular Design'' group at ZIB.

\subparagraph*{Nonreversible Processes}
%Schur Decomposition

The study of reversible processes is very advanced/well-established, based on eigendecomposition. %study/analysis. approach/idea
The idea to apply a Schur Decomposition instead has been proposed by Röblitz and further promoted by Weber. This generalized approach includes the special case of non-reversible processes and thus could become the generalized approach to analyze stochastic processes. %improved/enhanced/...

%Like it has been the case for reversible processes, the Schur decomposition approach could be stretched out to continuous processes, also including nonreversible transfer operators.

In general, the research of non-reversible processes is rather at the beginning. Using Schur Decomposition could yield many results for those processes.

Many processes occuring in real life are nonreversible (which ones?).

%As in real life, there are many nonreversible processes and, like in the example of the rebinding effect, which can only be reasonably bounded for nonreversible processes, the examination of such processes is crucial. The Schur Decomposition, as a generalization of the eigendecomposition, seems to be an appropriate approach to tackle this problem.
%Though it has to be developed more elaborated, like for continuous processes, as this topic has just begun to raise.
%It has been first proposed by Röblitz in her dissertation and been further advances by Weber (ZIB report).

\subparagraph*{Time-Dependent Processes: Coherent Sets}

\subparagraph*{``Rebinding Effect'' for other processes}
The Rebinding Effect, or Recrossing Effect, can occur if we project a process. Thus, rebinding or recrossing events can happen in all other kind of processes. What is the role of it for these processes? \marginpar{examples}
%in the way we presented in 1.3
For this reason, we brought up the topic of Raman spectroscopy.

%3 sehr aktuelle Themen in dieser Arbeit: Rebinding Effect, SchurDecompop of a process, Application: Raman Spectroscopy
