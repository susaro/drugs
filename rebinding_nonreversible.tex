\section{Approach for non-reversible Processes}
\label{sec:rebinding_nonreversible}

Employing the ideas presented in section \ref{sec:nonreversible}, we extend the problem of computing the minimal rebinding effect of a system to non-reversible processes. %tools/ideas. can be tackled
%(NESS processes). Djurdevac et al\cite{djur2016}(2016).

\subsubsection*{Schur Decomposition}
The transfer operator $\Pcal$ from \eqref{eq:transferoperator} describing a \textbf{non-reversible} process is \textbf{not} self-adjoint and hence, real eigenvalues are not guaranteed. %\ref{thm:selfadjoint_reversible}
In that case, problems as depicted in section \ref{sec:nonreversible} like complex eigenvectors can occur and consequently, we have to employ the Schur decomposition instead of the eigendecomposition in order to obtain a set of \textbf{real} vectors $X$ spanning an invariant subspace. % and being orthogonal. %compute/employ
%fuzzy clustering in terms of eigenvectors is in general not feasible/realizable/accomplishable
%In order to obtain a set of orthogonal vectors $X$ spanning an invariant subspace. fulfilling \eqref{eq:invariant} and \eqref{eq:orthogonal}
However, we cannot compute a real Schur decomposition for a continuous operator.
Therefore, we need to discretize the transfer operator $\Pcal$ to a matrix $P \in \R^{N \times N}$ at first. \marginpar{how?}
Then we can determine a subset $X \in \R^{N \times n}$ of real Schur vectors in order to compute the membership vectors as a linear combination $\chi = XA$ with a transformation matrix $A$, for instance obtained by Gen-PCCA. %orthogonal

As shown in section \ref{sec:nonreversible}, a reasonable result is achieved by employing Schur vectors that are orthogonal with respect to \textbf{any} initial distribution $\eta$.
%employing Schur vectors that are orthogonal with respect to \textbf{any} initial distribution $\eta$ yield a reasonable clustering.
However, we choose orthogonality with respect to the stationary distribution $\mu$. \marginpar{NESS?}

\subsubsection*{Galerkin Projection}

%Having the membership vectors $\chi = XA$, for instance obtained by Gen-PCCA+ or using any other feasible transformation matrix $A$, we can compute 
The clustered transition matrix is given by
%respectively its matrix decomposition $S\inv T$.
\begin{equation*}
P_c = S\inv T =  \langle \chi, \chi \rangle_\mu\inv \langle \chi, P \chi \rangle_\mu.
\end{equation*}
As demonstrated in the proof of theorem \ref{thm:iteration_error}, it can as well be represented by
%the projected process can be computed by
\begin{equation*}
P_c = A\inv \Lambda_s A,
\end{equation*}
where $\Lambda_s$ is the Schur matrix with the dominant eigenvalues and blocks. %sorted. consisting of
The corresponding transition rate matrix $Q_c$ can be computed by $Q_c = \frac{1}{\tau} \log(P_c)$ or equivalently by
\begin{equation}
\label{eq:schur}
Q_c = A\inv \Xi_s A,
\end{equation}
where $\Xi_s = \frac{1}{\tau} \log(\Lambda_s)$ is the Schur decomposition of the $Q_c$. %Schur block matrix

\subsubsection*{Rebinding Effect}

Again, we are interested in the rebinding effect included in a clustered system $Q_c$. %given by a transition rate matrix
If we know the utilized membership functions $\chi$ or the transformation matrix $A$, then we can measure the \textbf{real} rebinding effect which is encoded in the overlap matrix $S = D\inv A^T A$.
\\

%minimal rebinding effect: smallest rebinding effect under all possible transformation matrices A
However, if we want to estimate the \textbf{minimal} rebinding effect included in a system $Q_c$, then it is not sufficient to solve optimization problem \eqref{eq:optimization}. %since that was
It was based on the clustering with eigenvectors, assuming an originally reversible system, while the actual clustering is based on Schur vectors. In \eqref{eq:schur}, the invariant subspace $\Xi_s$ is not necessarily diagonal, but may have a block-diagonal shape. %/structure.
%. The current/actual clustering though 
%In the non-reversible case, we did not compute the membership functions based on orthogonal eigenvectors, but on \textbf{orthogonal Schur vectors}.
%Thus
%\begin{equation*}
%	Q_c = A\inv \Xi A
%\end{equation*}
%with Schur matrix $\Xi$ instead of Eigen-matrix.

Consequently, the columns of the transformation matrix $A$ are not multiples of the eigenvectors of $Q_c$, but of Schur vectors.
Thus, optimization problem \eqref{eq:optimization} has to be modified such that it is formulated in terms of Schur vectors: %modified/adapted
\begin{equation}
	\label{eq:schur_decomposition}
	Q_c X = X \Xi_s.
\end{equation}
Apart from that, %everything remains as in section \ref{sec:minimizing}; the side constraints are the same except for the replacement of eigenvectors by Schur vectors.
%In principle,
the optimization problem coincides with \eqref{eq:optimization}; we aim at minimizing the rebinding effect by maximizing $\det(S)$. The side constraints guarantee the necessary structure of $A$, leading to a stochastic matrix $S$. Though the columns of $A$ correspond to multiple of Schur vectors $X$ from \eqref{eq:schur_decomposition} now:

\begin{equation}
%\label{eq:optimization}
\mbox{
	\boxed{ \min_{\alpha_1, \dots, \alpha_n \in \R} |\det(S) -1|},
}
\end{equation}
%where we have to include several \textit{side constraints}. As the inverse matrix $A^{-1}$ consists of linear combinations of eigenvectors $X_i$, we have to consider
\begin{equation*}
	\mbox{
		\boxed{ \alpha_1 = 1 \ \ \ \mathrm{and} \ \ \ A_{ij}^{-1} = \alpha_j X_{ij} \ \ \forall i,j}.
	}
\end{equation*}
%Furthermore, $S$ is a stochastic matrix, see theorem \ref{thm:galerkin_stochastic}, and its structure is given in terms of the linear transformation matrix $A$, so we have two further constraints
\begin{equation*}
	\mbox{
		\boxed{S = D^{-1} A^TA \ \ \ \mathrm{and} \ \ \ S_{ij} \geq 0 \ \ \forall i,j}.
	}
\end{equation*}

\subsubsection*{Comparison}

%There exist several different cases which have to be distinguished. % and should not be confused!
%We have to distinguish between several cases.

In section \ref{sec:minimizing}, we examined originally \textbf{reversible} processes $Q$, clustered onto a subspace $Q_c$. In that case, the clustered process could be either reversible or non-reversible (not to confuse with a full decomposition clustering where reversibility is inherited).
The reversible (clustered) case is rather uninteresting as it yields the trivial solution as minimal rebinding effect. However, also non-reversible clustered matrices can be examined via the eigen-decomposition, since the eigenvalues are real and thus, no complex entries in the eigenvectors are possible.

%(for instance GenPCCA)
In contrast to that, an originally \textbf{non-reversible} process is clustered in terms of Schur vectors.
%In contrast to that, if the original process was \textbf{non-reversible}, then the clustering happened/took place in terms of Schur vectors.
Then, the initial point for the optimization problem is not given by the eigenvalue problem \eqref{eq:eigenvalue_problem}, but in terms of the Schur decomposition \eqref{eq:schur_decomposition},
%\begin{equation*}
%	Q_c X = X \Xi_s.
%\end{equation*}
%Here, we have to remark, that not only the $X$ are Schur vectors, but also $\Xi_s$ is not diagonal, but the Schur matrix,
with $\Xi_s$ possibly having outer diagonal elements.
%\newpage

%The clustered matrix is given by $Q_c = A\inv \Xi_s A$, and thus, the columns of $A\inv$ are multiples of the Schur vectors.

%But we don't have this useful relation $\exp(\xi_i) = \lambda_i$. What is the relation of $\Lambda_s$ and $\Xi_s$? (Block matrices)

%\subsubsection*{Advantage of Schur vectors}

The advantages of employing the Schur decomposition instead of the spectral decomposition have already been emphasized in section \ref{sec:nonreversible}. Also in this context, 
it is advantageous, since it generalizes the optimization problem, including reversible as well as non-reversible original systems.
%they yield some advantages, since they generalize the optimization problem, including reversible as well as non-reversible original processes.

%enables/allows
%The utilization of Schur vectors provides a generalization of the eigenvector-based approach. They include reversible as well as non-reversible processes.
%Hence, it is not necessary to know if the original process was reversible or not.

\subsubsection*{Motivation}

In which cases would it be interesting to compute the minimal rebinding effect? We can use it as an estimation (though bad!) for the real rebinding effect.
If we know the clustering, the real rebinding effect is calculated fast/easily.
Thus: interesting for systems which are clustered (or can be interpreted as a projection), where we don't know the original process or the clustering $(\chi,A)$.
\\

However, in that case it is not adequate to solve the previous optimization problem \eqref{eq:optimization}. If the original process was non-reversible, then the clustering has to be interpreted as based on Schur vectors instead of eigenvectors. % Hence the ansatz $Q_c X = X \Xi$ is wrong. Using this optimization problem would yield the wrong solution, assuming an original reversible system!
\\

%Solution: Cluster with Schur vectors (GenPCCA), get a projected process $Q_c = A\inv \Xi_s A$, where $\Xi_s$ is the (sorted!) block Schur matrix.
%\\

%Compare it with $Q_c X = X \Xi_s$, where $X$ are the (sorted!) Schur vectors. \marginpar{possible? Schur vectors not unique}
%Thus, as before, the columns of $A\inv$ can be interpreted as multiples of the Schur vectors. ...
%\\

Thus, we can compute the minimal rebinding effect for \textbf{any} system, independent of the reversibility/non-reversibility of the original process.
\\

Nevertheless, this estimation/bound may be good or not. But at least: applicable!

%A comparison of this optimization problem
The quality of this estimation will be evaluated in the next chapter by means of an exemplary reversible process, which is perturbed to non-reversibility.