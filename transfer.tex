\section{Transfer Operator}
\label{sec:transfer}

%trans. fct: prop. of distributions/ measures?
With the previously defined transition function, we have a tool to describe the propagation of \textbf{distributions} of stochastic processes.
\marginpar{= trans. matr. of MC?}
Now we introduce an operator that propagates \textbf{probability densities} of Markov processes. \marginpar{resp. sets?} %subensembles
Before defining such an operator, we have to specify the space of functions the operator is acting on.

\subsubsection*{$L^r$-Spaces}

It seems natural to define such a density propagating operator as acting on $L^1(\mu)$, the Banach space that includes all probability densities with respect to $\mu$. However it is sometimes advantageous to restrict the analysis to $L^2(\mu)$, since this can lead to a self-adjoint operator.
%justifications, reasons, motivations
As there are different motivations for the choice of a suitable space, we define an operator which acts on $L^r(\mu)$-spaces, i.e. spaces of $r$-integrable functions.

\begin{defi}[$L^r$-Spaces]
%prob. space?
Let $(E, \Sigma, \mu)$ a measure space. Then we define the corresponding $L^r$-spaces as equivalence classes of measurable functions
%\marginpar{$\mathbb{C}$?}
%quot spaces
\begin{equation*}
L^r(E, \Sigma, \mu) = \{ f: E \rightarrow \R \mid \int_E |f(x)|^r \mu(\diff x) < \infty \}
\end{equation*}
for $1 \leq r < \infty$ and
\begin{equation*}
L^\infty(E, \Sigma, \mu) = \{ f: E \rightarrow \R \mid \esssup_{x\in E} |f(x)|^r \mu(\diff x) < \infty \},
\end{equation*}
with the corresponding norms $\Vert \cdot \Vert_r$ and $\Vert \cdot \Vert_\infty$.
%definition of norms?
\end{defi}

In these equivalence classes, two functions $f,g$ are the identified if $f=g$ $\mu$-almost everywhere, see Werner\cite[section I.1]{werner2006funktionalanalysis}.
If it is clear from the context, which measure space $(E, \Sigma, \mu)$ is in consideration, we just write shortly $L^r(\mu) := L^r(E,\Sigma,\mu)$.
Due to H\"olders inequality, we have $L^r(\mu) \subset L^s(\mu)$ for all $1 \leq s \leq r \leq \infty$.
All $L^r$-spaces are Banach spaces, though $L^2(\mu)$ is the only one which can be equipped with a canonical scalar product and thereby becomes a Hilbert space, see Werner\cite[section V.1]{werner2006funktionalanalysis}. For $f,g \in L^2(\mu)$, the scalar product is defined as
\marginpar{$\mathbb{C}$?}
\begin{equation*}
\langle f,g \rangle_\mu := \int_E f(x) \widebar{g(x)} \mu(\diff x).
\end{equation*}
%initial distribution
Now let $\nu_0$ be the density function of a given start distribution.
%corresponding to the given start distribution (function).
Then the density function of a subset $A \in \Sigma$ at time $t$ is given in terms of the transition function by
%lag-time t?
\begin{equation*}
\nu_t(A) = \int_E \nu_0 p(t,x,A) \mu(\diff x).
\end{equation*}
On the other hand, the density $\nu_t$ is given by
\begin{equation*}
\nu_t(A) = \int_A \nu_t(x) \mu(\diff x).
\end{equation*}

\subsubsection*{Forward and Backward Transfer Operator}
%transport of functions; u.a. probability densitiy functions?
%result/yield in the following
The two above equations result in the following intuitive definition of a transfer operator which should ``propagate'' probability densities according to a given Markov process. But instead of limiting us to density functions, we define the transfer operator as acting on any $r$-integrable function.
\marginpar{propagates subensembles?}

\begin{defi}[Propagator or Forward Transfer Operator]
%Nielsen p.16
Let $p: \T \times E \times \Sigma \rightarrow [0,1]$ be the transition function of a Markov Process $\markov$ and $\mu$ be an invariant measure of $\markov$.
The semigroup of \textit{propagators} or \textit{forward transfer operators} $\Tcal^t : L^r (\mu) \rightarrow L^r (\mu)$ with $t \in \T$ and $1\leq r \leq \infty$ is defined via
\marginpar{semigroup?}
\begin{equation}
\label{eq:forward}
\int_A \Tcal^t \nu(y) \mu(\diff y) = \int_E \nu(x)p(t,x,A) \mu(\diff x)
\end{equation}
for all $A \in \Sigma$ and $\nu \in L^r(\mu)$.
\end{defi}
%\marginpar{anschauung?}
The propagator is well-defined on the Banach spaces $L^r(\mu)$, $1 \leq r \leq \infty$, see \cite{huisinga2001metastability}.
%announce her already some properties, list some properties
We will list already some properties of this operator which will be useful in the following chapters.
$\Tcal^t$ is a \textit{Markov operator}, i.e. it conserves the norm, $\Vert \Tcal^t\nu \Vert_1 = \Vert \nu \Vert_1$, and is positive, $\Tcal^t \nu \geq 0$ for $\nu \geq 0$.
%\marginpar{Markov operator? bounded?}
$\Tcal^t \nu_0$ describes the transport of the function $\nu_0$ in time $t$ by the underlying dynamics given by the process $X_t$ and weighted with respect to $\mu$ via
\begin{equation*}
\nu_0 \mapsto \nu_t = \Tcal^t \nu_0.
\end{equation*}
Since $\mu$ is invariant, we see immediately that the characteristic function $\eins := \eins_E$ of the entire state space is invariant under the action of $\Tcal^t$, that is \marginpar{?}
\begin{equation*}
\Tcal^t \eins = \eins.
\end{equation*}
It means that $\Tcal^t$ has the eigenvalue $1$ which corresponds to its eigenfunction $\eins$.


\begin{defi}[Backwards Transfer Operator\footnote{This nomenclature is motivated by the fact that for some models the forward transfer operator is related to the forward Kolmogorov relation, while the backward transfer operator is is related to the backward Kolmogorov relation.}]
The \textit{backwards transfer operator} $\Ucal^t: L^r(\mu) \rightarrow L^r(\mu)$ with $t \in \T$ and $1 \leq r  \leq \infty$ is defined by
\begin{equation}
\label{eq:backwards}
\Ucal^t f(x) = \int_E f(y) p(t,x,\diff y).
\end{equation}
\end{defi}
We have again $1$ as eigenvalue to the eigenfunction $\eins$, that is for all $t \in \T$ we have
%For all $t \in \T$ we have again
\marginpar{$\Vert \Ucal f \Vert \leq \Vert f \Vert$ $\Vert \Ucal \Vert \leq 1$ ?}
\begin{equation*}
\Ucal^t \eins = \eins.
\end{equation*}
The operator $\Ucal^t$ is \textit{adjoint} to $\Tcal^t$, denoted by $(\Tcal^t)^{*} = \Ucal^t$, that is they are related via the duality bracket, 
%the duality bracket $\langle f,g \rangle_\mu = \int_E f(x)g(x) \mu(\diff x)$,
namely for all $f \in L^p(\mu), g \in L^q(\mu)$ with $\frac1p + \frac 1q$, we have
%\marginpar{which $L$?}
\marginpar{$\Ucal := \Ucal^t$?}
\begin{equation*}
\langle \Tcal^t f, g \rangle_\mu = \langle f, \Ucal^t g \rangle_\mu.
\end{equation*}
We again remark that both forward as well as backward operator can be defined on arbitrary $L^r(\mu)$-spaces. But the previous equation shows us that either the choice $p=q=2$ or the choice $p=1, q= \infty$ or conversely, make sense, in order to obtain this useful adjointness/duality-relation of the two operators.

If we compare the equations \eqref{eq:forward} and \eqref{eq:backwards}, the notion of ``forward'' and ``backwards'' becomes clear. For the forward case, the state average with respect to $f$ is taken over all initial states $x$ which are propagated forward in time. In the backward case, we take the state average over all final states $y$.
\\

If the state space is finite and the corresponding process reversible, then we can see the relation of the forward and backward operator still better.
%Nielsen MA p.46
\marginpar{?}
Then the forward operator corresponds to the transition matrix, propagating probability distributions, while the backward operator corresponds to the transposed transition matrix, propagating subsets.

%\subsubsection*{Transfer operator of reversible processes}
\subsubsection*{Spectrum of Transfer operator}

Later in this thesis, we will be interested in examining the spectrum of the transfer operator of a given Markov process.
The following theorems give us an important insight about the spectrum and its relation to the reversibility of the process.
%corresponding process.
%Since self-adjointness of the transfer operator is equivalent to reversibility of the process, we will find out that only reversible processes guarantee a real spectrum.

\begin{defi}[Self-adjoint Operator]
An operator $\Tcal$ on $L^2(\mu)$ is called \textit{self-adjoint} if for all $f,g \in L^2(\mu)$ we have
\begin{equation*}
\langle f, \Tcal g \rangle_\mu = \langle \Tcal f, g \rangle_\mu.
\end{equation*}
\end{defi}

% corollary VII.1.2
\begin{thm}[Werner{\cite[theorem VI.1.2, theorem VI.1.3, lemma VI.3.1]{werner2006funktionalanalysis}}]
\label{thm:spectrum_operator}
Let $X$ be a Banach space and $\Tcal: X \rightarrow X$ a linear continuous operator. Then we have
%Then we can bound the absolute value of its eigenvalues by the operator norm
\begin{equation*}
| \lambda | \leq \Vert \Tcal \Vert \ \textrm{ for all } \ \lambda \in \sigma(\Tcal).
\end{equation*}
%additionally
If $X$ is a Hilbert space, then
\begin{enumerate}
\item $\sigma(\Tcal^{*}) = \{ \bar{\lambda} \mid \lambda \in \sigma(\Tcal)\}$,
\item if $\Tcal$ is self-adjoint, i.e. if $\Tcal^{*} = \Tcal$, then $\sigma(\Tcal) \subset \R$,
\item if $\Tcal$ is self-adjoint, then each two eigenfunctions corresponding to different eigenvalues are orthogonal.
\end{enumerate}
\end{thm}

Since we know that the operator norm of any transfer operator $\Tcal$ is $1$, it follows immediately from theorem \ref{thm:spectrum_operator} that its spectrum $\sigma(\Tcal)$ is contained in the unit circle of the complex plane, that is we have $| \lambda | \leq 1$ for all $\lambda \in \sigma(\Tcal) \subset \mathbb{C}$. %\marginpar{stat.meas. $\pi$}

\begin{thm}[Huisinga{\cite[proposition 1.1]{huisinga2001metastability}}]
\label{thm:selfadjoint_reversible}
Let $\Tcal^t: L^2(\mu) \subset L^1(\mu) \rightarrow L^2(\mu)$ be the propagator corresponding to the Markov process $\markov$. Then $\Tcal^t$ is self-adjoint with repect to the scalar product $\langle \cdot, \cdot \rangle_\mu$ in $L^2(\mu)$
if and only if $\markov$ is reversible.
\end{thm}

Thus, the transfer operator of a reversible process has a spectrum $\sigma(\Tcal) \subset [-1, 1]$.
%However, a nonreversible process can have complex eigenvalues. But as they are bounded by the operator norm, they have absolute values at most one, $|\lambda| \leq 1$. So they have to be contained in the unit circle in the complex plane.
Furthermore, theorem \ref{thm:spectrum_operator} guarantees us that the spectrum of a self-adjoint operator is equal to the spectrum of its adjoint. Thus, if we are given a reversible process, it doesn't matter if we examine the spectrum of the forward or the backward transfer operator.

\subsubsection*{Infinitesimal Generator}
%\marginpar{describes the rate a MC moves between states}

For $\T = \R$ the Chapman-Kolmogorov property \eqref{eq:chapman} of the transition functions makes the family $\{\Tcal^t\}_{t\in\R}$ of transfer operators a continuous \textit{semigroup} due to \marginpar{proof} \marginpar{also backw.?}
\begin{equation*}
\Tcal^{t+s} = \Tcal^t \Tcal^s.
\end{equation*}
This leads to the following definition of the (time-independent) infinitesimal generator.

\begin{defi}[Infinitesimal Generator]
For the semigroup of propagators or forward transfer operators $\Tcal^t : L^r (\mu) \rightarrow L^r (\mu)$ with $t \in \T$ and $1\leq r \leq \infty$ we define $\mathcal{D}(L)$ as the set of all $f \in L^r(\mu)$ s.t. the strong limit
\begin{equation*}
\Qcal f = \lim_{t \rightarrow 0} \frac{\Tcal^t f - f}{t}
\end{equation*}
exists. Then the operator $\Qcal: \mathcal{D}(L) \rightarrow L^r(\mu)$ is called the \textit{infinitesimal generator} corresponding to the semigroup $\Tcal^t$.
\end{defi}

The infinitesimal generator is an operator which describes the behaviour of a Markov process in infinitesimal time. That becomes clear by the relation
\begin{equation*}
\Tcal^t = \exp{(t \Qcal)}
\end{equation*}
in $L^2(\mu)$. \marginpar{ref} We say that $\Qcal$ ``generates'' the semigroup of transfer operators $\{\Tcal_t\}_{t \in \R}$ since the whole semi-group of transfer operators can be derived from it.

%So the spectrum of the infinitesimal generator is $\Lambda_1=0,\dots$.
%stationary distribution
%\marginpar{?} $\Qcal\pi = 0$;
If the corresponding Markov process is reversible, then $\Qcal$ is self-adjoint in $L^2(\mu)$ and consequentially the spectrum is contained in $(-\infty,0]$.
Therefore the dominant eigenvalues $1 = \lambda_1,\dots,\lambda_n$ of the propagator $\Tcal^t$ are related to the dominant eigenvalues $0 = \xi_1,\dots,\xi_n$ of the generator $\Qcal$ via %\marginpar{not yet discrete?}
%in the following way:
\begin{equation*}
	\lambda_k = \exp(t\xi_k)
\end{equation*}
for all $1\leq k \leq n$ and the associated eigenfunctions are identical. Thus the invariant measure $\eins$ of $\Tcal^t$ satisfies \marginpar{$0$ = fct.}%(dominant)
\begin{equation*}
	\Qcal\eins = 0.
\end{equation*}