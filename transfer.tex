\section{Transfer Operator}
\label{sec:transfer}

%trans. fct: prop. of distributions/ measures?
With the previously defined transition function, we have a tool to describe the propagation of \textbf{distributions} of stochastic processes. Now we are going to introduce an operator that propagates \textbf{probability densities} of Markov processes.

\subsubsection*{$L^r$-Spaces}

As probability densities are defined about their integral properties, \marginpar{no?}
%(integral = 1) (?),
we need some convenient integrable spaces on which the transfer operator can act.
%suitable
%we can define the transfer operator?
We are going to define an operator which acts just generally on $L^r$-spaces, i.e. $r$-integrable spaces.

\begin{defi}[$L^r$-Spaces] We define the $r$-integrable spaces as
\begin{equation*}
L_\mu^r(\X) = \{ f: \X \rightarrow \R \mid \int_\X |f(x)|^r \mu(\diff x) < \infty \}
\end{equation*}
for $1 \leq r < \infty$ and
\begin{equation*}
L_\mu^\infty(\X) = \{ f: \X \rightarrow \R \mid \esssup_{x\in\X} |f(x)|^r \mu(\diff x) < \infty \}.
\end{equation*}
\end{defi}
%If its clear which space and measure are considered, we just omit them and simplify the notation to
% a $L^r := L_\mu^r(\X)$
The space $L^2$ is the only of the $L^r$-spaces which can be equipped with a canonical scalar product and thus becomes a Hilbert space (see..). For $f,g \in L^2$ we define \marginpar{complex needed?}
\begin{equation*}
\langle f,g \rangle_\mu := \int_\X f(x)g(x) \mu(\diff x).
\end{equation*}
Now let $\nu_0$ be the density function of a given start distribution.
%corresponding to the given start distribution (function).
Then the density function of a subset $A \subset \X$ at time $t$ is given in terms of the transition function by
\begin{equation*}
\nu_t(A) = \int_\X \nu_0 p(t,x,A) \mu(\diff x).
\end{equation*}
On the other hand, the density $\nu_t$ is given by
\begin{equation*}
\nu_t(A) = \int_A \nu_t(x) \mu(\diff x).
\end{equation*}

\subsubsection*{Transfer Operator and spectral properties}
%transport of functions; u.a. probability densitiy functions?
The two equations above yield in the following intuitive definition of a transfer operator which should ``propagate'' probability densities according to a given Markov process. But instead of limiting us to density functions, we define the transfer operator as acting on any $r$-integrable function.
\marginpar{propagates subensembles?}

\begin{defi}[Transfer Operator] \marginpar{propagator}
%Nielsen p.16
Let $p: \T \times \X \times \B(\X) \rightarrow [0,1]$ be the transition function of a Markov Process $\markov$ and $\mu$ be an invariant measure of $\markov$.
The semigroup of \textit{propagators} or \textit{(forward) transfer operators} $\Tcal^t : L_\mu^r (\X) \rightarrow L_\mu^r (\X)$ with $t \in T$ and $1\leq r \leq \infty$ is defined via
\begin{equation}
\int_A \Tcal^t \nu(y) \mu(\diff y) = \int_\mathbb{X} \nu(x)p(t,x,A) \mu(\diff x)
\end{equation}
for all measurable $A \subset \mathbb{X}$ and $\nu \in L^r$. \marginpar{$A \in \Sigma$}
\end{defi}
%\marginpar{anschauung?}
The transfer operator is well-defined, see \cite{huisinga2001metastability}.
We will announce here already some properties of the transfer operator which will be useful in the following chapter(s).
$\Tcal^t$ is a \textit{Markov operator}; i.e. it conserves the norm, $\Vert \Tcal^t\nu \Vert_1 = \Vert \nu \Vert_1$, and is positive, $\Tcal^t \nu \geq 0$ for $\nu \geq 0$. \marginpar{Markov operator?}
$\Tcal^t \nu_0$ describes the transport of the function $\nu_0$ in time $t$ by the underlying dynamics given by the process $X_t$ and weighted with respect to $\mu$:
\begin{equation*}
\nu_0 \mapsto \nu_t = \Tcal^t \nu_0.
\end{equation*}
Since $\mu$ is invariant(?), we have that the characteristic function of the state space is invariant under the action of $\Tcal^t$, i.e.
\begin{equation*}
\Tcal^t \mathbb{1}_\X = \mathbb{1}_\X.
\end{equation*}
It means that $\Tcal^t$ has the eigenvalue $1$ which corresponds to its eigenfunction $\mathbb{1}_\X$.
\\

%since bblabla, we can deduce that
Furthermore $\Tcal$ is a bounded operator with operator norm $\Vert \Tcal \Vert_2 =1$ \marginpar{why only in $L^2$?} and $\Tcal  \mathbb{1}_\X = \mathbb{1}_\X$. This implies that the spectrum $\sigma(\Tcal)$ of $\Tcal$ is contained in the unit circle of the complex plane; i.e. we have $| \lambda | \leq 1$ for all $\lambda \in \sigma(\Tcal) \subset \mathbb{C}$. \marginpar{stat.meas. $\pi$}

\subsubsection*{Transfer operator of reversible processes}

The following two theorems give us an important insight about the spectrum of the transfer operator. Since self-adjointness of the transfer operator is equivalent to reversibility of the process, we know that only reversible processes guarantees a real spectrum!

\begin{defi}[Self-adjoint Operator]
An operator $\Tcal$ on $L^2(\mu)$ is called \textit{self-adjoint} if for all $f,g \in L_\mu^2(\X)$
\marginpar{def adj.}
\begin{equation*}
\langle f, \Tcal g \rangle_\mu = \langle \Tcal f, g \rangle_\mu.
\end{equation*}
\end{defi}

\begin{lem}
\label{thm:selfadjoint_real}
A self-adjoint operator has only real eigenvalues; $\sigma(\Tcal) \subset \R$.
\end{lem}
\begin{proof}
\end{proof}

\begin{thm}
\label{thm:selfadjoint_reversible}
Let $\Tcal^t: L_\mu^2(\X) \subset L_\mu^1(\X) \rightarrow L_\mu^2(\X)$ denote the propagator corresponding to the Markov process $X_t$. Then $\Tcal^t$ is self-adjoint with repect to the scalar product $\langle \cdot, \cdot \rangle_\mu$ in $L_\mu^2(\X)$
%, i.e.
%\begin{equation*}
%\langle u, \Tcal^t v \rangle_\mu = \langle \Tcal^t u, v \rangle_\mu \ \textrm{ for all } u,v \in L_\mu^2(\X),
%\end{equation*}
if and only if $X_t$ is reversible.
\end{thm}
\begin{proof}
Huisinga\cite{huisinga2001metastability}
\end{proof}
Since the spectral radius of any transfer operator is $1$, it follows from the previous two theorems that a reversible process has a spectrum $\sigma(\Tcal) \subset [-1, 1]$.

Later in this thesis, we are going to be very interested in examining the spectrum of the transfer operator of a given Markov process. Unfortunately we also have to consider nonreversible processes, so with a nonreal (complex) spectrum which will be a bit harder to compute with.

\subsubsection*{Backwards Operator}

\begin{defi}[Backwards Operator]
The \textit{backwards transfer operator} $\Ucal^t: L^r(\mu) \rightarrow L^r(\mu)$ with $t \in \T$ and $1 \leq r  \leq \infty$ is defined by
\begin{equation}
\Ucal f(x) = \int_\X f(y) p(t,x,\diff y).
\end{equation}
\end{defi}
For all $t \in \T$ we have again
\begin{equation*}
\Ucal^t \mathbb{1}_\X = \mathbb{1}_\X.
\end{equation*}
The operator $\Ucal$ is \textit{adjoint} to $\Tcal^t$, that is they are related via the duality bracket, namely
\begin{equation*}
\langle \Tcal^t f, g \rangle_\mu = \langle f, \Ucal^t g \rangle_\mu
\end{equation*}
for all $f,g \in L$. \marginpar{which $L$?}
Thus, if a Markov process $X_t$ is reversible, then we have
\begin{equation*}
\Tcal f(x) = \Ucal f(x)
\end{equation*}
for the corresponding forward and backward operators $\Tcal$ and $\Ucal$.

\subsubsection*{Infinitesimal Generator}
%\marginpar{describes the rate a MC moves between states}

For $\T = \R$ the Chapman-Kolmogorov property of the transition functions makes the family $\{\Tcal^t\}_{t\in\R}$ a continuous \textit{semigroup} due to \marginpar{proof}
\begin{equation*}
\Tcal^{t+s} = \Tcal^t \Tcal^s.
\end{equation*}
This leads to the following definition of the the infinitesimal generator. \marginpar{time-indep.?}

\begin{defi}[Infinitesimal Generator]
For the semigroup of propagators or forward transfer operators $\Tcal^t : L_\mu^r (\X) \rightarrow L_\mu^r (\X)$ with $t \in T$ and $1\leq r \leq \infty$ we define $\mathcal{D}(L)$ as the set of all $\nu \in L_\mu^r(\X)$ s.t. the strong limit
\begin{equation*}
\Qcal \nu = \lim_{t \rightarrow 0} \frac{\Tcal^t\nu-\nu}{t}.
\end{equation*}
exists. Then the operator $\Qcal: \mathcal{D}(L) \rightarrow L_\mu^r(\X)$ is called the \textit{infinitesimal generator} corresponding to the semigroup $\Tcal^t$.
\end{defi}

The infinitesimal generator is an operator which describes the behaviour of a Markov process in infinitesimal time. That becomes clear by the relation $\Tcal^t = \exp{(t \Qcal)}$ in $L^2$. \marginpar{ref} We can say that $\Qcal$ ``generates'' the semigroup of transfer operators since the whole semi-group of transfer operators can be derived from it.

Therefore, the eigenvalues $1 = \lambda_1,\dots,\lambda_m$ of the propagator $\Tcal^t$ are related to the eigenvalues $0 = \Lambda_1,\dots,\Lambda_m$ of the generator $\Qcal$ via \marginpar{not yet discrete?}
%in the following way:
\begin{equation*}
\lambda_k = \exp(t\Lambda_k)
\end{equation*}
for all $1\leq k \leq m$. Their corresponding (associated) eigenfunctions are identical.
%So the spectrum of the infinitesimal generator is $\Lambda_1=0,\dots$.
\marginpar{properties of generator}
Thus, the stationary distribution of $\Tcal^t$ is the solution of $\Qcal\pi = 0$;  $\Qcal\eins = 0$. \marginpar{?}