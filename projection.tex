\section{Molecular Kinetics as a Projection}
\label{sec:projection}
\marginpar{Mol. Kinetics Weber p.10}

% deduce/develop/derive two different point of views 
In this section, we will basically embed the mathematical concepts/results of chapter \ref{chap:markov} into a chemical/physical context in order to get a rigorous description of molecular (dynamic/kinetic?) systems. In considering such systems, we can distinguish between two point of views: we will see how we can get from the \textit{atomistic (=microsopic)} to a  \textit{macroscopic} scale/point of view by a projection.

\subsubsection*{Micro States}
%Boltzmann distribution = example for a canonical ensemble
A micro state of a molecular system with $N$ atoms can be represented in a $6N$-dimensional \textit{phase space} $\Gamma = \Omega \times \Rdrei$, \marginpar{spatial/position space}
consisting of the \textit{configurational space} $\Omega = \Rdrei$ and the \textit{momentum space} $\Rdrei$. In the following, we consider systems in \textit{thermodynamical equilibrium}.
%assume equilibrium
%\R_{+}
One possible model is given by the \textit{Boltzmann distribution} $\pi: \Omega \times \R^{3N} \rightarrow \R$, a
\marginpar{prob. dens. fct.?}
probability distribution assigning to each micro state a probability depending on its energy and temperature,
see McQuarrie\cite{mcquarrie2000}. It can be expressed as
% in the form
\begin{equation}
%\pi(q,p) \propto \exp{(-\beta H(q,p))},
\pi(q,p) = \frac{1}{Z} \exp{(-\beta H(q,p))},
\end{equation}
%The distribution shows that states with lower energy will always have a higher probability of being occupied %than the states with higher energy
where $\beta = 1/ (k_BT)$ is the inverse of the temperature $T$ multiplied with the Boltzmann constant $k_B$
%\marginpar{$\diff$?}
and $Z= \int_\Gamma \exp{-\beta H(q,p)} \diff(q,p)$ is the normalization factor. The Hamilton function denoted by $H$ is given by $H(q,p) = K(p)+V(q)$, the sum of the kinetic energy $K(p)$ and the potential energy $V(q)$.
Thus, the Boltzmann distribution $\pi$ can be decomposed into $\pi = \pi_p \pi_q$,
%as?
\begin{equation*}
%\pi(q,p) \propto \exp{(-\beta K(p))} \exp{(-\beta V(q))}, \underbrace{a}{b}
\pi(q,p) =  \underbrace{\frac{1}{Z_p} \exp{(-\beta K(p))}}_{\pi_p} \cdot
\underbrace{\frac{1}{Z_q} \exp{(-\beta V(q))}}_{\pi_q},
\end{equation*}
where $\pi_p: \R^{3N}  \rightarrow \R$ is the probability density function of the kinetic part in the momentum space $\R^{3N}$ and $\pi_q: \Omega \rightarrow \R$ is the probability density function of the potential part in the configurational space $\Omega$.


As we are interested in examining conformations/metastable sets, which are objects in configurational space, we will restrict ourselves to $\Omega$:
%Huisinga p.12
\begin{quote}
``A conformation $C \subset \Omega$ will be identified with the particular metastable sub-ensemble $\mu_{C \times \R^{3N}}$ corresponding to the particular subset $C \times \R^{3N} \subset \Gamma$. Hence, for every position $q \in C$, the conformation contains all states with $q \in \Omega$ and arbitrary $p \in \R^{3N}$.''
\end{quote}
In this sense, conformations/metastable sets contain no information on momenta and are determined in configurational space only. We are considering a reduced model in position space with a \textit{reduced density} $\pi_q = \int_{\R^{3N}} \pi(q,p) \diff p$. \marginpar{?}

\todo{difference conformation vs. metastable set}
%Later, we will project our dynamics onto $\Omega$;

\subsubsection*{Macro States via Membership Functions}

As the phase space and even the configurational space are very large, we aim to reveal the underlying  discrete Markov State Model by  group/cluster a collection of the micro states having the same or similar values in one observable. Such a collection of micro states will be called a \textit{macro state}.
For instance, that could be the states/observables ``bound'' or ``unbound'' for a receptor-ligand system.
\marginpar{entropic inf.?}
%and entropic information?
%Macro states need not be distinct sets.

We apply the function-based clustering method presented in section \ref{sec:fuzzy}. \marginpar{overlap = good?}
We define macro states as overlapping partial densities, which can be identified as membership functions $\cfam$.
%using membership functions which can have certain overlap.
The membership functions $\chi_1,\dots,\chi_n : \Omega \rightarrow [0,1]$ form a partition of unity, i.e.
\begin{equation}
\label{eq:statistical_weights}
\sum_{i=1}^n \chi_i(q) = 1.
\end{equation}
%So: membership function "$=$" macro state? NOP
By grouping micro states, the (corresponding) macro states yield \textit{statistical weights}
%We can assign a statistical weight to each macro state (= membership fct. $\chi_i$):
\marginpar{what is that good for?}
\marginpar{$e$ vs. $\eins$}
\begin{equation*}
%pi_q?
%here: continuous process and thus \eins instead of e
w_i = \langle \chi_i, \eins \rangle_\pi = \int_\Omega \chi_i(q) \pi_q(q) \diff q.
\end{equation*}
The statistical weight $w_i$ corresponds to the ``probability to be in conformation $\chi_i$''.
%(metastable) macro state $i$

\subsubsection*{Transfer Operator}
%Weber habil p.14
Each micro state $(q,p) \in \Gamma$ determines a \textit{probability density function} $\Psi^{-\tau}(\cdot \mid (q,p))$ describing the possible evolutions of the system in configurational space $\Omega$ in time $\tau$, being started at the initial state $(q,p)$.
%For instance, $\Psi^{-\tau}(\tilde{q} \mid (q,p))$ is the probability to get to $\tilde{q} \in \Omega$ starting %from $(q,p) \in \Gamma$.
%For a given micro state $(q,p) \in \Gamma$, we define the probability density function $\Psi^{-\tau}%(\tilde{q} \mid (q,p))$ as the probability of the system to get to $\tilde{q} \in \Omega$ in time %(step) $\tau$, after being started in $(q,p)$.
Weber\cite{weber2011subspace} defines a transfer operator $\Pcal (\tau): L^{1,2}_{\pi_q}(\Omega) \rightarrow L^{1,2}_{\pi_q}(\Omega)$ for the propagation of (membership) functions via \marginpar{why not densities?}
\begin{equation}
\label{eq:transferoperator}
\Pcal(\tau)f(q) = \int_\Rdrei \left( \int_\Omega f(\tilde{q})\Psi^{-r}(\tilde{q}\mid (q,p)) \diff \tilde{q} \right) \pi_p(p)\diff p.
\end{equation}
In this definition, the density function $\Psi^{-\tau}(\cdot \mid (q,p))$ can be interpreted as a transition function as defined in section \ref{sec:markov}.
We have to notice that this transfer operator corresponds to the \textit{backward operator} from section \ref{sec:transfer}. \marginpar{?}
%We have to notice that this transfer operator doesn't correspond to the transfer operator from section %\ref{sec:transfer}, since that one was defined to be a \textit{forward} operator. However/in contrast, the %transfer operator \eqref{eq:transferoperator} is a \textit{backward} operator.

It is a \textit{generalized} transfer operator in the sense that it includes deterministic as well as stochastic dynamical models. In order to describe deterministic dynamics, the density function $\Psi^{-\tau}$ has to be chosen as a Dirac delta function, since an initial state $(q(0),p(0))$ determines exactly the future states in configurational space.
\\

%Weber Habil p.28 Markov Operator!
%\marginpar{adjoint operator}
% (corresponding)
It is important to remark that the transfer operator $\Pcal(\tau)$ also defines a
\marginpar{propagator sec \ref{sec:transfer}}
projected \textit{Markov operator} \marginpar{def} $\overline{\Pcal} (\tau)$ acting in configurational space $\Omega$, see Weber\cite{weber2011subspace}, by
\begin{equation}
\label{eq:markov_operator}
\overline{\Pcal} (\tau) = \pi_q \circ \Pcal(\tau) \circ (\pi_q)^{-1},
\end{equation}
which propagates density functions.
The previous equation shows that the space of membership functions is connected to the space of density functions by multiplication with $\pi_q$.
%Weber Habil.
We will keep that relation in mind, but just use $\Pcal$ in the following.
%Pcal = propagates sets/membership fct; Pcalbar = propagates densities.
\\

As $\Pcal(\tau)$ in \eqref{eq:transferoperator} propagates \textbf{membership functions}, stationarity is characterized by the equation $e=\Pcal(\tau)e$ for the constant function $e = 1$ in $\Omega$. For the Markov operator $\overline{\Pcal} (\tau) $ in \eqref{eq:markov_operator} propagating \textbf{densities}, stationarity can be characterized by $\pi$ = $\overline{\Pcal} (\tau) \pi$, where $\pi$ is the Boltzmann density. \marginpar{Boltzmann dens. = dist.?}
These two operators are \textit{adjoint} operators. This can also be seen by the fact that a discretization of $\Pcal(\tau)$ results in a matrix $P_c$, while a discretization of $\overline{\Pcal} (\tau)$ will result in the transposed matrix $P_c^T$.

\subsubsection*{Maybe: Properties of transfer operator for reversible Processes}
%\subsubsection*{Spectrum of Transfer Operator for reversible processes}
Detailed Balance
\begin{equation}
\label{eq:detailed_balance}
\pi_q (\tilde{q}) \cdot \int_{\R^{3N}} \Psi^{-r}(q \mid (\tilde{q},p)) \pi_p(p)\diff p 
	= \pi_q(q) \cdot \int_{\R^{3N}} \Psi^{-r}(\tilde{q}\mid (q,p)) \pi_p(p)\diff p
\end{equation}

\subsubsection*{Markov State Model for reversible Processes}

%equation/condition
For now, we consider a reversible process. Then due to the detailed balance condition \eqref{eq:detailed_balance}, the corresponding transfer operator $\Pcal$ is \textbf{self-adjoint} and thus has a real spectrum, see theorem \ref{thm:selfadjoint_real} (follows from linearity and self-adjointness) and $\sigma(\Pcal) \subset [-1,1]$ (since $\Vert \Pcal f \Vert_{\pi_q} \leq \Vert f \Vert_{\pi_q}$).
\marginpar{self-adj. wrt $\pi_q$}
%corresponding process
%If the process is non-reversible, then this operator is not self-adjoint and hence can possess complex %eigenvalues. We will handle this case in section \ref{sec:rebinding_nonreversible}.
In order to apply the spectral approach from section \ref{sec:spectral}, we assume that the \textbf{discrete spectrum} of the transfer operator $\Pcal$ has $n$ \textbf{dominant eigenvalues}
%Why? -> metastable sets
$1 = \lambda_1 \geq \lambda_2 \geq \dots \geq \lambda_n$ which are all close to $1$ and bounded away from the essential spectrum.
%see section \ref{sec:spectral}. %(see Sch\"utte?).
The corresponding dominant eigenfunctions are denoted by $\Xcal = \{ \Xcal_1,\dots, \Xcal_n\}$ and therefore the eigenvalue problem is $\Pcal (\tau) \Xcal = \Xcal \Lambda$, with the eigenvalue matrix $\Lambda = \mathrm{diag}(\lambda_1, \dots, \lambda_n)$.
\\

As we have seen in chapter \ref{chap:meta}, the number of metastable sets of a process can be determined by the number of dominant eigenvalues; i.e. we are going to create a Markov State Model on $n$ states.
%Since each of the $n$ dominant eigenvalues of the transfer operator corresponds to a metastable set (no? %because membership fct instead of set?)...
%Using the dominant spectrum of the transfer operator, we want to create a discrete Markov State Model on %$n$ states.
The state space of this model should consist of the macro states of our Molecular System and its transition behaviour should be described via a $n\times n$-transition matrix $P(\tau)$. \marginpar{?}
%(i.e. row-stochastic matrix).
%Each state space of this model shall be a macro state
In order to get from our continuous operator $\Pcal(\tau)$ to a discrete matrix $P(\tau)$, we need at first to determine the size and shape of the membership functions  $\chi_i$.
%We want to get from our continuous operator $\Pcal(\tau)$ to a discrete matrix $P(\tau)$, while preserving %the most important properties of the process. \marginpar{Markovianity?}
%Of course by reducing the dimension we will lose some of the original informations but the discrete model
%should be as good as possible
%At first to determine the size and shape of the membership function $\chi_i$.
As described in section \ref{sec:fuzzy}, this can be done by computing a linear combination of the dominant eigenfunctions via
%membership functions = always linear combination of eigenfunctions????
\begin{equation}
\label{eq:pcca_membership}
\chi_j(q) = \sum_{i=1}^N A_{ij}\Xcal_i(q), \ j=1,\dots,n,
\end{equation}
where $A= \{A_{ij}\}_{i,j=1,\dots,n}$ is the solution of PCCA+ (convex maximization problem).
\marginpar{PCCA+ only for finite/ discrete state spaces?}
% state space = finite (6N states, but on continuous values)
This choice of membership functions preserves Markovianity of the process when projecting. \marginpar{Ref?}
%onto finite subspace consisting of states chi1,..chin (membership fct= state??)
As a linear combination of eigenfunctions, the membership functions $\chi_i$ might have an overlap; they are not orthogonal!

\subsubsection*{Galerkin Projection}
Having computed the membership functions $\chi_i$, we can project $\Pcal(\tau)$ to a low-dimensional Markov State Model $P_c(\tau)$
%reduce our continuous stochastic process to a finite process
by the Galerkin discretization
%finite, discrete process/ MSM. Galerkin projection/discretization
\begin{equation}
\label{eq:galerkin}
P_c(\tau) = G(\Pcal(\tau)) = (\langle \chi, \chi \rangle_\pi)\inv (\langle \chi, \Pcal(\tau) \chi \rangle_\pi).
\end{equation}
%compare theorem \ref{thm:galerkin}.
%We can see that \eqref{eq:galerkin} fulfills Theorem \ref{thm:galerkin} in the case of set-based %conformations
%$\chi_i$, because then we have $\chi_i^2 = 1$ and $\chi_i\chi_j = 0$ for $i \neq j$ (indicator functions).
%as described in section \ref{sec:galerkin}.
In order to know about the quality of this model, we are interested in the iteration error under this projection. As mentioned in section \ref{sec:recrossing}, this error is zero if the Galerkin discretization of $(\Pcal(\tau))^k$ is equal to the iteration $(P_c(\tau))^k$. In that case, the diagram \ref{fig:diagram_transfer} commutes.
The following theorem shows that there is no discretization error under the projection \eqref{eq:galerkin}, i.e.  we have $(\Pcal(\tau))^k = (P(\tau))^k$, which implies that Markovianity is preserved. \marginpar{?}

\begin{thm}[Weber {\cite[Theorem 2]{weber2011subspace}}]
Let $\Pcal(\tau)$ be the $\pi_q$-self-adjoint transfer operator defined in \eqref{eq:transferoperator} with a set $\Xcal = \{ \Xcal_1,\dots, \Xcal_n\}$ of normalized eigenfunctions s.t. $\Pcal (\tau) \Xcal = \Xcal \Lambda$ and a set of functions $\chi = \Xcal A$ that is a linear combination of the eigenfunctions $\Xcal$ with a regular $n \times n$-transformation matrix $A$ from \eqref{eq:pcca_membership}.
Then the iteration error for the Galerkin discretization $P_c(\tau) = G(\Pcal(\tau))$ in \eqref{eq:galerkin} vanishes.
\end{thm}
\begin{proof}
%weber p.32
\end{proof}
It follows that the above projection represents the correct dynamical long-time behaviour of the original process and that the matrix $P_c(\tau)$ is the correct Markov State Model.
%For the projected transfer operator
%For the Markov State Model
We can use the matrix representation $P_c(\tau)=S^{-1}T$ from theorem \ref{thm:galerkin}. Then $S$ and $T$ are stochastic matrices with \marginpar{?}
%where $S$ is the matrix correponding to the statistical weights $w_i$.
\begin{equation}
\label{eq:projection_presentation}
\begin{aligned}
T & = D^{-1} \langle \chi, \Pcal(\tau) \chi \rangle_\pi = D^{-1}A^T \Lambda A \ \ \textrm{ and } \\
S & = D^{-1} \langle \chi, \chi \rangle_\pi = D^{-1} A^T A,
\end{aligned}
\end{equation}
%\begin{equation}
%T = D^{-1} \langle \chi, \Pcal(\tau) \chi \rangle_\pi = D^{-1}A^T \Lambda A \ \textrm{ and } \ 
%S= D^{-1} \langle \chi, \chi \rangle_\pi = D^{-1} A^T A,
%\end{equation}
where $D= \mathrm{diag}(w_1,\dots,w_n)$ is the diagonal matrix of statistical weights in \eqref{eq:statistical_weights}.

\subsubsection*{Measuring the Rebinding Effect} \marginpar{Interpretation}

%habil p.34
Even though $P_c(\tau) := P(\tau)$ is the correct Markov State Model, it cannot be interpreted as a transition matrix, since the inverse matrix of $S$ is not necessarily stochastic.
The matrix $T = D^{-1} \langle \chi, \Pcal \chi \rangle$ however can be interpreted as a transition matrix. Then the difference between $P_c(\tau)$ and $T$ is given by
\begin{equation*}
S P_c(\tau) = T.
\end{equation*}
Thus, the ``disturbance'' of ... can be measured by the matrix $S$.
%Interpretation of $T$ and $S$.
The more the matrix $S$ differs from the identity matrix, the more the correct projection $P(\tau)$ differs from the transition matrix $T$. Thus, the rebinding effect can be measured by the matrix $S$.
The trace of $S$ is at most $n$. Optimizing $\mathrm{trace}(S)$ is equivalent to optimizing the \textit{crispness} of the conformations $\chi$ (Röblitz).

%The data we get for our computations is based on simulations. As a transition rate matrix can be detected %from these simulations (instead of transition matrix), we formulate the above result for transition rates %instead of transition probabilities.

\subsubsection*{Infinitesimal Generator to transition rate matrix}

Often (...) it is more convenient to consider/examine/investigate transition rate matrices instead of transition matrices/ infinitesimal generators instead of transfer operators. We can define the same/similar/analogous Galerkin Projection on the corresponding infinitesimal generator.

Conceptually, $\Qcal$ is connected to the computation of transition rates.

The transfer operator $\Pcal(\tau)$ defines a time-independent operator $\Qcal$ via
\begin{equation*}
\Qcal = \lim_{\tau \rightarrow 0} \frac{\Pcal(\tau)-\mathcal{I}}{\tau},
\end{equation*}
which is the infinitesimal generator of $\Pcal$: \marginpar{Chapman}
\begin{equation*}
\Pcal(\tau) = \exp{(\tau\Qcal)}.
\end{equation*}
Weber\cite{weber2011subspace} shows that such an infinitesimal generator exists for a discretization in terms of membership functions.

Since the eigenfunctions of $\Qcal$ and $\Pcal$ are the same and their eigenvalues are related via $\exp{(\xi_i)} = \lambda_i$, we can apply the same Galerkin Projection for the infinitesimal generator as for the transfer operator in \eqref{eq:galerkin}.
%For the infinitesimal generator we can apply the same Galerkin Projection as for the transfer operator in
%\eqref{eq:galerkin}
We get a $n\times n$-rate matrix
\begin{equation}
\label{eq:galerkin_infinitesimal}
Q_c = A^{-1} \Xi A = (\langle \chi, \chi \rangle_\pi)\inv (\langle \chi, \Qcal \chi \rangle_\pi),
\end{equation}
where $\Xi$ is the diagonal matrix consisting of the $n$ leading eigenvalues $0 = \xi_1 > \xi_2 \geq \cdots \geq \xi_n$ of $\Qcal$ and $A$ is the transformation matrix of \eqref{eq:pcca_membership}, which analoguously transforms the eigenfunctions of $\Qcal$ into membership functions of the macro states.

The matrix $Q_c$ can be interpreted as a transition rate matrix. \marginpar{?}
\pagebreak