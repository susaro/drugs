In this chapter we are going to examine a special type of molecular dynamic systems, namely receptor-ligand systems.
To describe these systems rigorously we can use all the mathematical concepts/objects defined in the previous chapters.

To give a short overview about what is going to happen here. The kinetics of a molecular system can be described via a differential equation. The solution of this \marginpar{nop} differential equation is a Markov(?) process which can be described via a transfer operator (section \ref{sec:transfer}). This operator will be projected onto a finite-dimensional state space (section \ref{sec:galerkin}) which may spoil the Markov Property of the process (section \ref{sec:recrossing}). \marginpar{optimization}
%complies with
%mathematically based
This chapter is mainly based on Weber and Fackeldey\cite{weber2014}.
%also\cite{weber2012}?

%Here we will try to tackle this subject for nonreversible (NESS) processes also.

\todo{which model? Hamiltonian ($\Gamma$)? Hamiltonian w/ randomized momentum ($\Omega$ w/ any momentum)? Langevin? Diffusion Dynamics?}
%\todo{Ensemble}

\section{Receptor-Ligand System}

We will present the \textit{receptor-ligand-system} as particular molecular dynamic system and describe it mathematically using a differential equation.
We will discuss the so called \textit{Rebinding Effect} and set in in relation to the Recrossing effect known from section \ref{sec:recrossing}.
%merged/set together
%outlook for later/further investigations
As an outlook/motivation, we will explain how these two concepts can be set together in the important application of \textit{drug design} and how the rebinding effect can help to improve the effect/efficiency of drugs.
%an important application of the rebinding effect in such a receptor-ligand-system.

\subsubsection*{What is a MD system? Molecular Dynamics vs Kinetics} \marginpar{Weber p.10}
In the previous chapters, we have often mentioned molecular dynamic systems (MD systems) and some of their properties without actually explaining what such a system is.
A \textit{molecular system} consists of atoms that are connected by \textit{covalent bonds}. bla bla.
The potential energy function, or \textit{energy landscape}, of a molecular system results in a dynamical behaviour on different timescales.
The fastest timescales (vibrations of covalent bonds) are around $10^{-15}$ seconds. \marginpar{see ..}
bla. up to nanoseconds or, for protein folding, up to microseconds or seconds or even longer..

\subsubsection*{What is a Receptor-Ligand system?}

A receptor can be a protein or blabla. As we are mainly interested in the mathematical description of a system, we will not specify further, which kind of biochemic/molecular receptor we are considering. Instead for us, a receptor will just be an \textit{object} to which a ligand can bind.
%and thus prevents the receptor to bind to any other object/ligand. Ligand takes a degree of freedom away

A \textit{receptor} is (in general) a proteine molecule.
A molecule that binds to a receptor is called a \textit{ligand}.
Each receptor will only bind with ligands of a particular structure.
%lock-key

Ligand binding is an (chemical) equilibrium process, i.e. the reaction rates of the forward and backward reactions are equal. That means that the concentrations of the reactants and the products are constant
%doesn't change
(\textit{dynamic equilibrium}). \marginpar{see ..}

A ligand (L) can bind to a receptor (R) and form a receptor-ligand complex (LR) which can dissolve again into its two original components. These reactions can be represented in the following form
%can be described by the 2-state process
\begin{equation*}
\label{eq:reaction}
L + R \rightleftharpoons LR,
\end{equation*}
%representing the law of mass action?
which corresponds to the law of mass action. \marginpar{see ..}
The dissociation constant $k_d$
%is an equilibrium constant that ...: \marginpar{what is that?}
\begin{equation*}
k_d = \frac{[L] \cdot [R]}{[LR]},
\end{equation*}
where $[L], [R]$ and $[LR]$ are the concentrations of $(L),(R)$ and $(LR)$, respectively. This constant is commonly used to describe the affinity between a ligand $(L)$ and a protein $(P)$, i.e. how strongly/tightly the ligand can bind to his particular protein/receptor. If the dissociation constant is small, then there is a high binding affinity between the ligand and the receptor.
%Affinity is a measure of the tendency of a ligand to bind to its receptor. Efficacy is the measure of the bound %ligand to activate its receptor.
%Affinity: The ability of a drug to combine with a receptor to create a drug-receptor complex.
%Efficacy: The ability of a drug-receptor complex to initiate a response.
The association constant $k_a$ is just the inverse of the dissociation constant
\begin{equation*}
k_a = \frac{[LR]}{[L] \cdot [R]}.
\end{equation*}
There are different factors which can lead to a high or low \textit{binding affinity}. \marginpar{which?}
%like thermodynamical reasons, ... .

\subsubsection*{Bivalent Ligand}

Bivalent ligands consists of two (drug-like?) molecules connected by an (inert?) linker.

\subsubsection*{Mathematical Description of Receptor-Ligand-System}

Starting from the reaction equation \eqref{eq:reaction}, we can deduce that the ligand can be found in two different (macro) states: ``unbound'' $(L)$ or ``bound'' $(LR)$. Then the probabilities of the ligand to be in one of these states can be described by the probability vector $x^T = \frac{1}{s}(([L],[LR]))$, where $s = [L] + [LR] = \textrm{const.}$.
This leads to an ordinary differential equation
\begin{equation*}
\dot{x}^T = x^T Q_c.
\end{equation*}
The matrix $Q_c$ consists of the rates of reaction,
\begin{equation*}
Q_c = 
\begin{pmatrix}
-k_a[R] & k_a[R]  \\
k_d      & -k_d
\end{pmatrix},
\end{equation*}
where $k_a$ and $k_d$ are the association and dissociation constants. Thus, it is the transition rate matrix corresponding to a Markov chain, i.e. it describes a memoryless process.

We will later see that this mathematical description of a receptor-ligand-system is not accurate, since in fact, such a process \textit{will} have some kind of memory.

%\section{Rebinding Effect and Drug Design}
\marginpar{impact of multivalency on rebinding effect (Weber, Chem.)}

\subsubsection*{Rebinding Effect}

The rebinding effect has been characterized as a memory effect which leads to an additional thermodynamic weight of the bound state.
%Weber quantifying rebinding effect
%occurs when projecting a MD process onto a finite subspace??

In fact, a stochastic process describing a receptor-ligand molecular system is NOT necessarily Markovian. The Markovianity can be spoiled by the Rebinding Effect. If a Receptor-Ligand system dissolves, due to the favorable spatial situation (?) it is more likely to rebind again than to stay dissolved. %\marginpar{Zusammenhang zum Absatz davor?}

There are several papers (...) describing the rebinding effect from a chemical and a mathematical point of view. In chemistry, there are several reasons/factors for the rebinding effect discussed.

The rebinding effect has recently (...) been discussed to increase the binding affinity of a process.
\marginpar{?}

\subsubsection*{Relation to Recrossing Effect}
We remember that we described the recrossing effect in section \ref{sec:recrossing}. There, we also had a process on macro states described by a transition matrix and thus being a Markov chain. But in reality, the (clustered) process was not memoryless.
The same phenomena occurs with the rebinding effect. We have a transition matrix, even though our process has a memory. We want to quantify this effect.

\subsubsection*{Basics of Drug Design}
%inventive process of finding new medications
The term \textit{drug design} describes the development of new medications based on the knowledge of a
%Most commonly,
biological target. The drug is often a small molecule which can bind to a protein molecule (target/disease/receptor) and thus activates or inhibits its function (disease modifying). So drug design is basically about designing a molecule which is
%complementary to the binding site of target
complementary in shape and charge to the biomolecular target and therefore will bind to it, see Str{\o}mgaard et al\cite{stromgaard2002}.
%is about developing
More precisely, drug design describes the design of ligands, i.e. molecules that will bind tightly to the given target, see Tollenaere\cite{tollenaere1996}.
%target = receptor; drug = ligand.. simplified model
%disease: have a bad molecule/protein/receptor which can bind to a human cell and create sickness.
%in order to avoid sickness: want a drug/medicament/ molecule which binds to the disease, s.t.
%it is not possible to bind to the human anymore :)

%The most fundamental goal in drug design is to predict whether a given molecule will bind to a target and if %so how strongly.
%tendency to bind

In order to be an efficient drug, we aim/wish for a a high \textit{binding affinity}, which is a measure of the strength of the chemical bond.
That means that the designed ligands should easily bind to the receptors, remain in a binding or rebind quickly after being dissolved. If the binding affinity of a drug is too low, a higher concentration of the drug is needed instead,
%in order to be effective,
which is undesired because of possible side effects.
There are many factors that influence/affect the binding affinity of a drug/ligand, such as ... .
The rebinding effect has been recently investigated to increase the binding affinity of a ligand, mathematically described by Weber et al\cite{weber2012, weber2014} as well as chemically by e.g. Vauqelin\cite{vauquelin2010}. This effect will be examined in this thesis. As a \textit{high} rebinding effect is aimed/wished, we will derive a lower bound for this effect, i.e. we will \textit{minimize} it.
%several factors for high binding affinity: like good shape, multivalency

\subsubsection*{Drug Design}

An important application of receptor-ligand processes is drug design. In short: A drug consists of ligands which should bind to the receptors of the virus. If the drug creates many bindings, the virus is "bound" and cannot attack the human (cell?) anymore. Thus, many bindings are a favorable thing. So a high rebinding effect enhances the (overall?) binding affinity of the process/ system which is good for the efficiency of a drug. We want a high rebinding effect.
So in this chapter, we examine the minimal rebinding effect for a given Kinetics.
This task has been solved by Weber and Fackeldey\cite{weber2014} for reversible processes.
\marginpar{How: from bound to unbound}