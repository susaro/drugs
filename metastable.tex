%method/technique
With the Galerkin discretization, we introduced a method to reduce the dimension of a Markov process by projecting it onto a smaller state space.
However we don't know yet how to choose a partition of unity such that this projection yields a reasonable Markov State Model, in the sense that important properties of the original process are maintained.
%inherited,maintained
%by metastability, by a metastable behaviour, metastable sets
Usually, the long-time behaviour of a process is of particular interest. It is often determined by a so called metastable behaviour.
%It describes the typical behaviour of rare transitions between specific subsets, called \textit{metastable} or \textit{dominant subsets}, after a long duration of stay inside.
We will see why it makes sense to project a process onto its metastable sets and, in order to detect them, analyze their relation to the dominant spectrum of the transfer operator.
We will also see that the optimal metastable decomposition is not sharp/crisp but soft/fuzzy.
%in terms of membership functions

%introduce/present. decompose the state space/cluster a process/create a MSM
Additionally, with the aid of the Schur decomposition we introduce a rather new approach to create a Markov State Model. In contrast to the spectral approach, it includes nonreversible processes and is even able to identify further structures.
%cluster a broader class of processes.


\section{Metastability}
\label{sec:metastability}

There exist several definitions of metastability. %different definitions
%Shortly said, metastability is the property of a process that its state space consists of subsets/regions such that transitions between these subsets are rare events while the duration of stay inside of each of them is rather long.
Shortly said, metastability is the property of a process to act on particular regions such that transitions between these regions are rare events while the durations of stay inside of each of them is rather long.
%these subsets is rather long.
Some possible characterizations of that behaviour are based on large hitting times or small exit rates, see Sch\"utte and Sarich\cite[chapter 3]{schutte2013metastability},
where a good overview of the most common definitions can be found.
%which gives a good overview of the most common definitions.

\subsubsection*{Mathematical concept of metastability}

%define, introduce. $A \in \Sigma$?
In order to describe the concept of metastability, it is a good way to start with so called \textit{stable} or \textit{invariant subsets}. A measurable subset $A \subset E$ of the state space of a Markov process $X_t$ is called stable or invariant if it cannot be left, i.e. if $\Prob(X_t \in A \mid X_0 \in A) = 1$ for all $t$.
%Similarly
Analoguously, we can define a \textit{metastable} or \textit{almost invariant subset} as a subset in which the process will stay for a very long time before exiting it into any other subset, that is $\Prob(X_{t_f} \in A \mid X_0 \in A) \approx 1$ for a convenient timescale $t_f$.
%Metastability: A subset is not invariant but almost invariant
%Consequentially
Thus, a full partition $A_1,\dots,A_n$ of the state space $E$ is called \textit{metastable} if
\begin{equation}
\label{eq:metastability}
\sum_{k=1}^n \Prob_\mu(X_{t_f} \in A_k \mid X_0 \in A_k) \approx n.
\end{equation}
%mu??
Then each of the sets $A_k$ is almost invariant with respect to timescale $t_f$;
the probability to stay in one of the partition sets being started there is almost $1$, while the probability to change between any two different partition sets is almost $0$.
Such a partition is also called a \textit{metastable decomposition}.
%of the state space?

Obviously, being ``close to 1'' or ``close to $n$'' are rather vague statements.
However, that lack of concreteness will be eliminated later, since we will only be interested in the ``best'' metastable decomposition.
%That means that we want to obtain a decomposition where the sum \eqref{eq:metastability} is as close as possible to $m$, or equivalently the probability to stay inside of a metastable set is as close as possible to $1$.
%High holding probability?
That means that we want to obtain a decomposition where the probability to stay inside of each metastable set is as close as possible to $1$, resulting in the sum \eqref{eq:metastability} being as close as possible to $n$.
Likewise, the choice of the timescale $t_f$ is not specified in general and will depend on the particular system in consideration.
%has to be specified for each model individually. system/question in consideration/investigation/case
Hence, the only parameter in \eqref{eq:metastability} that has to be determined is the number $n$ of subsets we are looking for.
%\marginpar{joint metastability?}

\subsubsection*{Metastability in molecular systems}

Metastability is a very important concept for stochastic processes corresponding to molecular systems. Such processes describe the movement of atoms or molecules in space and
have the characteristic behaviour to oscillate or fluctuate around equilibrium positions on the smallest time scales (about one femtosecond). \marginpar{BM?}
In contrast to these fast oscillations, the process often stays inside of a certain region, called \textit{conformation}, for a long time before switching to another region (nano- or millisecond time scale).
%%switching/changing/exiting. region/subset. comparatively/relatively
Since transitions between conformations are relatively rare events, they can be identified as metastable sets if we choose a convenient timescale.
Such a behaviour is depicted in figure \ref{fig:conformations} on the example of the dihedral angle of a molecule, taken from Weber\cite{weber2011subspace}.

%\marginpar{conformation = spatial arrangement? special case of metastability?}
% long duration of stay + small transition rates/probabilities/ rare transitions between these subsets

\begin{figure}[!htb]
	\label{fig:conformations}
	\centering
	\includegraphics[width=0.7\textwidth]{conformations.jpg} %70% der Textbreite
	\caption{Example of a molecule with two (metastable) conformations. The dihedral angle of the molecule can take values between $+45^\circ$ and $-45^\circ$. There are two regions, highlighted red and blue, where the process stays for a rather long time and oscillates \textbf{inside}. Transition between these regions don't occur often. Thus, these two conformations can be identified as metastable sets.}
\end{figure}


%This dihedral angle can take values between $+45^\circ$ and $-45^\circ$.
%Thus, the process acts on a continuous state space and has infinitely many states.
%There are two regions, highlighted red and blue, where the process stays for a rather long time and oscillates \textbf{inside}.
%Transition between these regions don't occur often.
%and that transitions between these regions are rare.
%Thus, these two conformations can be identified as metastable sets.
%(metastable) conformations.

%useful
As transitions between metastable sets are rare events, long-time simulations of a process are required in order to get informations about conformational changes.
%%Thus, conformational changes are rare events and will show up only in long term simulations of the dynamics (nano- or millisecond time scale).
%complex systems
However long-time simulations of such large systems are not feasible in reasonable time even with the best computers nowaday, see Anton\cite{shaw2009millisecond} or its successor Anton$2$\cite{shaw2014anton}.
%supercomputers, designed for the purpose of such molecular dynamic simulations. %Unfortunately..

%needed/required
Hence, in order to be able to compute some long-time simulations of a given molecular system, a reduction of complexity  is needed. This can be achieved by a clustering of the state space via a Galerkin projection as depicted in section \ref{sec:galerkin}. Different states will be clustered appropriately such that we get a process on a smaller state space.
%depicted, described
\\

This point of view also motivates the following terminology. A state in the original state space is called a \textit{micro state}, as it is a state considered on the microscopic or atomistic level. In order to get a smaller state space, micro states are grouped together and such a cluster is called a \textit{macro state}, since we are now considering the process on a macroscopic level (cannot distinguish between smaller states/atoms anymore).

For instance, the spatial coordinates of a single atom could be considered as a micro state, while the corresponding macro state is a cluster of several atoms. If we are working on this smaller state space, we cannot distinguish anymore between the single atoms of the cluster (forget information).

\subsubsection*{Clustering into metastable sets}

%(i.e. how to choose the partition of unity for Galerkin projection)
%represents the correct long-time behaviour of the original process
The question \textbf{how} to cluster a given process such that it maintains the long-time behaviour of the original process can be answered easily with the following intuitive approach: As the long-time behaviour of a process is described by the metastability of the system, we choose the metastable sets as clustering sets.
%described, determined, based
More clearly, we create a new process where each macro state corresponds to one of the metastable sets.
In order to represent the correct long-time behaviour of the original process, the transition probabilities of the clustered process/reduced model should correspond to the transition rates between the metastable sets.
\\

%clusters/macro states/membership fcts.
%projected model

%cluster/group together
%As we are mainly interested in the long-time behaviour of a given process, it seems reasonable to cluster states of a metastable set together and create a new process where each macro state corresponds to one of the metastable sets. The transition probabilities of the clustered process/reduced model should correspond to the transition rates of the original process between its metastable sets. \marginpar{micro/macro states}
%between the macro states
%In order to..
%maintains, inherits, keeps
%determined in terms of
As metastability is determined on
%As metastability describes a behaviour on
%reduced model/projected process/clustered process
long timescales, the projected process maintains the long-time behaviour of the original process, but forgets about short-time transitions, i.e. transitions inside of a conformation/metastable set.

Since there is not one unique metastable decomposition of the state space, we need to find a decomposition which is in some sense ``the best''; then we can use it to create a reduced model. In the next sections we will see how to find such a decomposition.
\\

%clustering not unique: different metastable decompositions
%For example: in which metastable set should we assign a
%transition region (e.g. a region which is close to several metastable sets). SOLVED BY MEMBERSHIP?

%Going from micro states to macro states.

%\subsubsection*{Advantages / Disadvantages}

%projected/clustered
%reduced dimension/complexity
Most importantly, the clustered process will have the desired property of a reduced complexity since the model acts on a smaller state space.
%while maintaining the crucial property of the original process (transitions between metastable sets = long-time behaviour).
Therefore, the computation effort for long-time simulations is definitely decreased.
Furthermore, we get a better overview of the system, since it is always easier to consider a process on a few states in comparison to a process on a very large or even continuous state space.
However, it has to be guaranteed that the clustered process represents the \textbf{correct} long-time behaviour of the process. How this can be ensured will be explained in section \ref{sec:fuzzy}.
\clearpage