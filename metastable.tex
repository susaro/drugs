In section \ref{sec:galerkin}, we introduced a method/technique to reduce the dimension of a Markov process by projecting it onto a smaller state space.
But we don't know yet \textit{how} to choose a partition of unity such that the corresponding Galerkin projection yields a \textit{reasonable} Markov State Model, in the sense that important properties of the process are maintained.
The answer to the question,  what an \textit{important property} is, can differ from case to case.
For our investigations (and many other research in MD systems), we are particulary interested in the long-time behaviour of a process, which can be characterized by the concept of \textit{metastability}.
It describes the typical behaviour of rare transitions between specific subsets, called \textit{metastable} or \textit{dominant subsets}, after a long duration of stay inside.
%aforementioned
We will see why it makes sense to project the process onto its metastable sets and how they are related to the spectrum of the transfer operator.
%can be detected employing the spectrum of the transfer operator.
We will also see that the optimal metastable decomposition is not sharp/crisp but soft/fuzzy using \textit{membership functions}.
%using membership functions, in terms of membership functions
%membership fct. = part. of unity? Here: yes. Röblitz: required that each function takes value 1

For a nonreversible process, such a dominant structure can sometimes be given in terms of \textit{dominant cycles} instead of \textit{dominant sets}. We will introduce these structures as well and give a short outlook/overview how they can be detected.
%handled/tackled

%The basics of cycle representations stem from Kalpazidou\cite{kalpazidou2007} and its application to %nonreversible Markov processes from Djurdjevac, Weber and Sch\"utte\cite{djur2016}.

\section{Metastability}
\label{sec:metastability}

There exist several different definitions of metastability. Shortly said, metastability is the property that the state space of a process
%can be for arbitrary stoch. proc.?
%which has subsets/regions
consists of subsets/regions s.t. transitions between these subsets are rare events while the duration of stay inside of each of these subsets is rather long.
%(a decomposition into?)
%stays for a long time in certain subsets before moving to other subsets. \marginpar{schlecht beschrieben}
Some possible characterizations of that behaviour are based on large hitting times or small exit rates, see Sch\"utte and Sarich\cite[chapter 3]{schutte2013metastability} which gives a good overview of the most common definitions.
%relation/comparison

\subsubsection*{Mathematical concept of metastability}

%define, introduce. In order to..
In order to describe the concept of metastability, it is a good way to start with so called \textit{stable} or \textit{invariant subsets}. A subset $A \subset \X$ of the state space of a Markov process $X_t$ is called stable or invariant if it cannot be left, i.e. if $\Prob(X_t \in A \mid X_0 \in A) = 1$ for all $t$.
%Similarly
%$A \subset \X$
Analoguously, we can define a \textit{metastable} or \textit{almost invariant subset} as a subset in which the process will stay for a very long time before exiting it into any other subset; that is $\Prob(X_{t_f} \in A \mid X_0 \in A) \approx 1$ for a convenient timescale $t_f$.
%Metastability: A subset is not invariant but almost invariant
%Consequentially
Thus, a full partition $A_1,\dots,A_m$ of the state space $\X$ is called \textit{metastable} if
\begin{equation}
\label{eq:metastability}
\sum_{k=1}^m \Prob_\mu(X_{t_f} \in A_k \mid X_0 \in A_k) \approx m.
\end{equation}
%mu??
Then each of the sets $A_k$ is almost invariant with respect to timescale $t_f$;
the probability to stay in one of the partition sets being started there is almost $1$, while the probability to change between any two different partition sets is almost $0$.
Such a partition is also called a \textit{metastable decomposition}.
%of the state space?

Obviously, being ``close to 1'' or ``close to $m$'' are rather vague statements. But that lack of concreteness will be eliminated later,
%However,
since we will only be interested in the ``best'' metastable decomposition.
That means that we want to obtain a decomposition where the sum \eqref{eq:metastability} is as close as possible to $m$, or equivalently the probability to stay inside of a metastable set is as close as possible to $1$.
%High holding probability?
%That means that we want to obtain a decomposition where the probability to stay inside of a metastable set is as close as possible to $1$, resulting in the sum \eqref{eq:metastability} being as close as possible to $m$.
%so for each metastable set we try do get that probability as close to 1 as possible.
Also the choice of the timescale $t_f$ is not specified in general and will depend on the particular system in consideration.
%system/question in consideration/investigation.
%case.
Hence, the only parameter in \eqref{eq:metastability} that has to be determined is the number $m$ of subsets we are looking for. \marginpar{def. conformation?}
%determined before
\\

\marginpar{membership fct, cluster, metastability, macro state}
In section \ref{sec:galerkin}, we defined the Galerkin projection on an arbitrary partition of unity $\{\cfam\}$, instead of defining it just on full partitions. Such a partition of unity (membership fcts.) defines a projected transfer operator $P_c$.
Then the trace of $P_c$ is referred to as \textit{metastability} of the conformations $\{\cfam\}$.
%why that makes sense? high diagonal elements -> high eigenvalues -> high metastability
%projected transfer operator from theorem...
This definition of metastability corresponds to the full-partition formulation \eqref{eq:metastability}, since the projected process has the state space $\{\cfam\}$. Therefore, the diagonal of $P_c$ consists of the holding probabilities of the conformations.

\subsubsection*{Metastability in Molecular Dynamic Systems}
%\marginpar{results in need for clustering}

Metastability is a very important concept for stochastic processes describing molecular dynamic systems.
%a realization
Such processes often have the characteristic behaviour that their trajectory stays inside of a certain
%switching/changing/exiting. region/subset
region, also called \textit{conformation},  for a long time before switching to another region. Furthermore, transitions between these conformations are rare. \marginpar{conformation = spatial arrangement? special case of metastability?}
%So this behaviour
%So these regions
So they correpond to our above definition of metastable sets if we choose a convenient timescale.
% long duration of stay + small transition rates/probabilities/ rare transitions between these subsets

\begin{figure}[!htb]
	\label{fig:conformations}
	\centering
	\includegraphics[width=0.7\textwidth]{conformations.jpg} %70% der Textbreite
	\caption{Example of a molecule with two (metastable) conformations}
\end{figure}

%examplified
This behaviour is depicted in figure \ref{fig:conformations}, taken from ... . This example shows a molecule, whose dihedral angle can take values between $+45°$ and $-45°$, thus the process consists orginally of infinitely many states.
But we can see, that there are two regions (red and blue) where the process stays for a rather long time and that transitions between these regions are rare. Thus, they correspond to our characterization of metastable sets.
\\

%\subsubsection*{Problem of long-time simulations of MD systems}

%Metastability -> long Verweildauer in conformation -> long-time simulations required
%Long-time simulations -> not feasible with nowaday computers -> need reduction of dimension

%Because of that fact, rather long simulations of the system are required if we want to know more about %these "conformation changes".

As transitions between metastable sets are a rather rare event, we need to make long-time simulations of our process in order to get reasonable results about these changes of conformations.
But, as mentioned in Chapter \ref{chap:markov}, long-time simulations of such complex systems are not feasible even with the best computers nowaday, see (...).
%Unfortunately..

Thus, in order to be able to compute some long-time simulations of a given MD system, a reduction of complexity  is needed/required. This can be achieved by a clustering of the state space as described in section \ref{sec:galerkin}. Different states of the state space will be clustered appropriately s.t. we get a process on a smaller state space.

\subsubsection*{Clustering into metastable sets}

So far, we didn't mention how we should choose the clusters/macro states/membership fcts. for the Galerkin projection, i.e. wich states of the original process should be grouped together for the reduced model. %\marginpar{Galerkin = clustering?}
%projected model

%cluster/group together
As we are mainly interested in the long-time behaviour of a given process, it seems reasonable to cluster states of a metastable set together and create a new process where each (macro) state corresponds to one of the metastable sets. The transition probabilities of the clustered process/reduced model should correspond to the transition rates of the original process between its metastable sets. \marginpar{micro/macro states}
%between the macro states
%In order to..
%maintains..
As metastability describes a behaviour on long timescales, our newly created process should maintain the long-time behaviour of the original process, but \textit{forget} about its short-time transitions, i.e. transitions inside of a conformation/metastable set.

Since there is not one unique metastable decomposition of the state space, we need to find a decomposition which is in some sense ``the best''; then we can use it to create a reduced model. In sections \ref{sec:spectral} and \ref{sec:fuzzy} we will see how to find such a decomposition.

%clustering not unique: different metastable decompositions
%For example: in which metastable set should we assign a
%transition region (e.g. a region which is close to several metastable sets). SOLVED BY MEMBERSHIP?

%Going from micro states to macro states.

\subsubsection*{Advantages / Disadvantages} \marginpar{weg?}

Most importantly, our newly created process will have the desired property of a reduced dimension/complexity since the model acts on a smaller state space while maintaining the crucial property of the original process (transitions between metastable sets = long-time behaviour). So the computation effort for (long-time) simulations is definitely decreased.
Furthermore, we get a better overview of our process, since it is always easier to consider a process on a few states in comparison to a process on a very large or even continuous state space. Since fast/short-time transitions (transitions inside conformation/metastable set) are not our research goal, we just omit these (at least for our case!) superfluous informations.
But there is also a disadvantage, as already mentioned in section \ref{sec:recrossing}, by projecting a process it can lose its Markov property.
%\marginpar{under certain/ which conditions not?} WOANDERS!
