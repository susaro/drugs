\label{chap:meta}
%\section{Dominant metastable sets (reversible)}

In section \ref{sec:galerkin}, we already mentioned that under certain conditions (high complexity/ long-time simulations) a reduction of dimension of our process is needed. In order to be able to apply a reasonable Galerkin Projection, we are now going to introduce the concept of metastability which is a certain behaviour of the trajectory of  stochastic processes that occurs in many molecular dynamic systems \marginpar{see ..}
%not only Markov processes?
and describes the characteristics of rare transitions between specific subsets after a long duration of stay inside.
%case
We will see why it makes sense to choose these \textit{metastable sets}  as clustering sets in the aforementioned Galerkin Projection
%We will also see the relation of metastable sets and the spectrum of the transfer operator
and how they can be detected using the spectrum of the corresponding transfer operator.
We will also see that the optimal metastable decomposition is not shark/crisp but soft/fuzzy.
%(i.e. eigenvalues and eigenfunctions)
%which metastable decomposition is optimal (no unique solution) (PCCA+).
%which solution is optimal
%of the transfer operator of the corresponding process.

Unfortunately, metastability of nonreversible processes is much harder to detect, since we are not guaranteed a real spectrum of the transfer operator. So we need the concept of dominant cycles instead of dominant sets. We will give a short outlook how that case could be handled.
%tackled

%The basics of cycle representations stem from Kalpazidou\cite{kalpazidou2007} and its application to %nonreversible Markov processes from Djurdjevac, Weber and Sch\"utte\cite{djur2016}.

\section{Concept of metastability applied to MD}
\label{sec:metastability}

%views on metastability
There exist several different definitions of metastability. Shortly said, metastability is the property of a Markov
%can be for arbitrary stoch. proc.?
process which consists of (a decomposition into?) subsets s.t. transitions between these subsets are rare events while the duration of stay inside of each of these subsets is long.
%stays for a long time in certain subsets before moving to other subsets. \marginpar{schlecht beschrieben}
%For example, it can be characterized by large hitting times (see ...) or small exit rates (see ...).
Some possible characterizations of that behaviour are based on large hitting times (see...) or small exit rates (...).
%All of them give a good idea about the concept, so we will present some of them to make clear why %metastability is such an important concept, especially in molecular dynamics.
%for MD processes

\subsubsection*{Mathematical concept of metastability}
%and relation to MD

To describe the concept of metastability, it is a good way to start with so called \textit{stable} or \textit{invariant subsets}. A subset of the state space of a Markov process is called stable or invariant if it cannot be left, i.e. if $\Prob(X_t \in A \mid X_0 \in A) = 1$ for all $t$.
Analoguously, we can define a \textit{metastable} or \textit{almost invariant subset} as a subset in which the process will stay for a very long time before exiting it into any other subset; i.e. $\Prob(X_{t_f} \in A \mid X_0 \in A) \approx 1$ for a convenient timescale $t_f$.
%Metastability: A subset is not invariant but almost invariant
\\

%lack of concreteness can be eliminated ..
Obviously, being "close to 1" is a rather vague statement. But that lack of concreteness will be eliminated later,
%However,
since we will only be interested in the "best" metastable decomposition, so for each metastable set we try do get that probability as close to 1 as possible. Instead, we will have to determine the number of subsets we are looking for.
\\

Let $A_1,\dots,A_m$ be a full partition of the state space $\X$. Then this partition and each of its partition sets are called \textit{metastable} if
\begin{equation*}
\sum_{k=1}^m \Prob_\mu(X_{t_f} \in A_k \mid X_0 \in A_k) \approx m.
\end{equation*}
Then each of the sets $A_k$ is almost invariant with respect to timescale $t_f$;
the probability to stay in one of the partition sets being started there is almost $1$, while the probability to change between any two different partition sets is almost $0$.
%\begin{equation*}
%p(A_k,A_k) \approx 1 \textrm{ for all } k=1,\dots,m \textrm{ and } p(A_k,A_l) \approx 0  \textrm{ for all } %k,l=1,\dots,m \textrm{ s.t. } k \neq l
%\end{equation*}
\\

Have projected transfer operator $P$ by Galerkin projection with $\{\cfam\}$. Then the trace of $P$ is referred to as metastability of the conformations $\{\cfam\}$.

%\subsubsection*{Metastability in Molecular Dynamic Systems}

\subsubsection*{Importance of metastability to MD systems}

Metastability is a concept which is very important for stochastic processes describing molecular dynamic systems.
Such systems/processes often have the characteristic behaviour that their trajectory stays inside of a certain conformation/region/subset for a very long time before switching/changing/exiting to another region. Furthermore, transitions between these conformations are rare. 
%So this behaviour
So these regions correpond to our above definition of metastable sets if we choose a convenient timescale.
% long duration of stay + small transition rates/probabilities/ rare transitions between these subsets

\subsubsection*{Problem of long-time simulations of MD systems}

%Metastability -> long Verweildauer in conformation -> long-time simulations required
%Long-time simulations -> not feasible with nowaday computers -> need reduction of dimension

%Because of that fact, rather long simulations of the system are required if we want to know more about %these "conformation changes".

As transitions between metastable sets are a rather rare event, we need to make long-time simulations of our process in order to get reasonable results about these changes of conformations.
But, as mentioned in Chapter 2, long-time simulations of such complex systems are not feasible even with the best computers nowaday, see (...).
%Unfortunately..

%\subsubsection*{Solution: Galerkin Projection}
So, in order to be able to compute some long-time simulations of a given MD system, a reduction of complexity like described in section 2.4 is needed/ required. This can be achieved by a clustering of a state space. Different states of the state space will be clustered appropriately s.t. we get a process on a smaller state space.

\subsubsection*{Clustering into metastable sets}

So far, we didn't mention how we should choose the clusters for the Galerkin projection, i.e. wich states of the original states should be grouped together for the reduced/projected model. \marginpar{Galerkin = clustering???}

As we are mainly interested (for the moment) in the long-time behaviour of our process, we are not interested in short-time transitions, i.e. transitions inside of a conformation/ metastable set. So knowing now the concept of metastability, it seems directly/intuitively appropriate/reasonable to group/cluster states of a metastable together and create a new process where each state corresponds to one of the metastable sets.

Since we want to maintain/keep the long-time-behaviour/properties of the original space/system/process, the transition probabilities of the clustered/reduced process/model should correspond to the transition rates of the original process between its metastable sets.
In doing to, we get a new process which possesses the long-time-behaviour of the original process, but "forgets" its short-time-behaviour, i.e. transitions inside of a conformation.

%But such a clustering is not a eindeutig thing. For example: in which metastable set should we assign a %transition region (e.g. a region which is close to several metastable sets).

%The reduction of dimension (deletion of unnecessary information) gives us a much smaller state space than %before. Going from micro states to macro states.

%Summary:
%The metastability of a process is a reason for the need of too long and complex computations/ simulations. %But this problem can also be solved using metastability. Namely reducing the system onto metastable sets.


%\subsubsection*{Clustering and Why and How}
%Aim: complexity reduction

\subsubsection*{Advantages / Disadvantages}

First of all, our newly created process will have the desired property of a reduced dimension/complexity since the model acts on a (much) smaller state space while maintaining its crucial property (transitions between metastable sets = long-time behavious). So the computation effort for (long-time) simulations is definitely decreased.
Furthermore, we get a better overview of our process, since it is always easier to consider a process on a few states in comparison to a process on a very large or even continuous state space. Since fast/ short-time transitions (transitions inside conformation/ metastable set) are not our research goal, we just omit this (at least for our case!) superfluous informations.
%Not every value is interesting, but just conformations/ metastable clusters/ sets are interesting.
So our advantages in clustering are
\begin{enumerate}
\item reduction of dimension (main reason b/c otherwise computation are just not possible)
\item better overview of the crucial behaviour of the process (rare transitions)
%(nice side effect, also helpful)
\end{enumerate}
%Eigentlicher Grund: Reduction of Complexicity
%Guter Side-Effekt: Smaller number of states = anschaulicheres Model, besserer Überblich
%main factor: too many states -> not possible to compute with computers nowaday
%maybe also: not every of the maybe infinitely many states is of relevance: clustering to get better overview
But there is also a disadvantage, as we mentioned in 2.4, by projecting a process it can lose its Markov property. \marginpar{under certain/ which conditions not?} \marginpar{Disadv: no good metastability -> no good model?}

\subsubsection*{Example}

To explain metastability in Molecular Dynamics, consider the following easy example, figure ... .
Our system consists of ..? molecule and can take ...? values.
As we can see from figure ..., showing a long-time-simulation of that process,

\begin{figure}
	\centering
	\includegraphics[width=0.7\textwidth]{conformations.jpg} %70% der Textbreite
	\caption{Example of a molecule with two conformations}
\end{figure}
