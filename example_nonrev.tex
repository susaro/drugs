%\section{Outlook for nonreversible systems} %nonreversible Q
\section{Rebinding in a non-reversible System}

%examine different kinds of nonreversibility
%In order to find out if the rebinding effect in the clustering of a \textbf{non-reversible} system behaves differently than in the reversible case,
%we come back/further examine
In order to compare the rebinding effect in the clustering of a \textbf{non-reversible} system to the reversible case, we further examine the example from section \ref{sec:example_reversible}. We modify it slightly by introducing small \textbf{perturbations} in the eigendecomposition of the reversible process, leading to non-reversibility, as proposed by Weber\cite{Weber2017}.
%and thereby changing it to a Schur decomposition
%In order to find out the role of nonreversibility to $S_\textrm{opt}$, we follow the approach of Weber\cite{Weber2017} to classify the non-reversibility of a process by the real Schur decomposition of the transition matrix. It can be written as $P = X_s \Lambda_s X_s^{-1}$, where the matrix $X_s$ consists of the Schur vectors and the matrix $\Lambda_s$ is an upper triangular matrix with possible $2 \times 2$-blocks on its diagonal.
%For this aim, we examine the given Schur decomposition from Weber\cite{Weber2017}
%, but compute the corresponding transition rate matrices, instead of transition matrices.
The outcome shall be a Schur decomposition given by
\begin{equation}
\label{eq:schur_decomp}
\Lambda_s =
\begin{pmatrix}
1 & 0 & 0 & 0 & 0 \\
0 & 0.99 & \epsilon & 0 & 0 \\
0 & -\gamma & 0.98+\delta & 0 & 0 \\
0 & 0 & 0 & 0.005 & 0 \\
0 & 0 & 0 & 0 & 0.001
\end{pmatrix}.
\end{equation}
We compute the corresponding transition rate matrix by $P = X \Lambda_s X\inv$ with the Schur vectors $X$ being equal to the eigenvectors from \eqref{eq:transition_matrix}.
If $\Lambda_s$ is a diagonal matrix, then this equation represents the eigenvalue problem of a reversible process $P$. By introducing non-zero values for $\epsilon, \gamma$ and $\delta$, the system gets non-reversible.

Why is this example interesting? We examine different systems/matrices, having the same Schur vectors and rather similar Schur decompositions. However, these small changes in the Schur decomposition lead to different results when it comes to compute the rebinding effect.
%Normally det(S_real) chould be the same for all these systems, since tranformation matrices (feasibility) depends only on the eigen/Schur vectors

Accordingly to the example from section \ref{sec:example_reversible}, we compute the real rebinding effect, when projecting the process onto a three-dimensional subspace represented by the transition rate matrix $Q_c$ and compare it to the minimal rebinding effect included in that subspace.
Starting from the projected process $Q_c$, we use the Schur vectors $X$ from $Q_c X = X \Xi_s$, where $\Xi_s$ is the Schur decomposition with \textbf{sorted} Schur values.
%Schur matrix, consisting of the \textbf{sorted} Schur blocks.
%to solve optimization problem \eqref{eq:optimization}.
Since $\Xi_s$ has a $2 \times 2$ block on its diagonal, we cannot utilize optimization problem \eqref{eq:optimization} to compute the minimal rebinding effect.
Instead, we employ the same approach based on Schur vectors as described in section \ref{sec:rebinding_nonreversible}.
%estimate rebinding effect

\subsubsection*{$P$ is reversible}

$P$ is reversible if we set $\epsilon = \gamma = \delta = 0$ in the Schur decomposition \eqref{eq:schur_decomp}. In that case, the Schur vectors are also eigenvectors of $P$ and the system is equal to the one from section \ref{sec:example_reversible}. Therefore, we should obtain the same results when solving the optimization problem based on Schur vectors instead of eigenvectors.
%employing Schur vectors instead of eigenvectors.
In order to verify that, we compute again $200$ clusterings with random feasible transformation matrices. The result, depicted in appendix A.1 coincides with the result from section \ref{sec:example_reversible}.
%represented/presented. coincides/conformes/resembles

%From section \ref{sec:minimizing}, we know that the minimal rebinding effect is given by $\detSopt = 1$.

%However the application of PCCA+ yields an overlap matrix
%\begin{equation*}
%	\Sreal = \begin{pmatrix}
%		0.8677  &  0.0061  &  0.1262 \\
%		0.0058  &  0.9101  &  0.0841 \\
%		0.1144  &  0.0803  &  0.8053
%	\end{pmatrix},
%\end{equation*}
%which results in the ``real'' rebinding effect $\detSreal = 0.6171$.   Wo kommt das her?? PCCA+ macht orthogonale Matrix ohne overlap, det(S) = 1 ???????
%realized by application of PCCA+

\subsubsection*{$P$ is nonreversible with real eigenvalues}

If we set $\epsilon = 0.004$, the matrix $P$ becomes nonreversible, but still has real eigenvalues, since $\epsilon$ is on the upper triangle of $\Lambda_s$, while all eigenvalues are distinct (same algebraic and geometric dimension).

\begin{equation*}
	P = 
	\begin{pmatrix}
		0.9886  &  0.0011  	&  0.0011  	&  0.0046	& 0.0046 	\\
		0.0001 	&  0.4973  	&  0.4949  	&  0.0052  	& 0.0025	\\
		0.0001  &  0.4949	&  0.4973  	&  0.0025  	& 0.0052	\\
		0.0086  &  0.0018 	&  0.0004	&  0.4964  	& 0.4928	\\
		0.0086  &  0.0004	&  0.0018   &  0.4928  	& 0.4964
	\end{pmatrix}.
\end{equation*}

\begin{figure}[ht!]
	%	\label{fig:reb_example_nonrev}
	\centering
	\subfigure[The minimal rebinding effect compared to the degree of non-reversibility of the clustered system $Q_c$.]{\includegraphics[width=0.46\textwidth]{figures/rebinding/200_random_schur/epsilon0.004/reb_opt_nonrev_200_random.eps}}
	\subfigure[The minimal and the real rebinding effect compared to the degree of non-reversibility of the clustered system $Q_c$.]{\includegraphics[width=0.46\textwidth]{figures/rebinding/200_random_schur/epsilon0.004/reb_nonrev_200_random.eps}}
	%, increasing the probability of a fast \textbf{rebinding}.
	\hspace{20pt}
	\subfigure[The minimal rebinding effect $\det(\Sopt)$ compared to the real rebinding effect $\det(\Sreal)$ included in $Q_c$.]{\includegraphics[width=0.46\textwidth]{figures/rebinding/200_random_schur/epsilon0.004/reb_200_random.eps}}
	\caption{The system described by the transition matrix $P$ is clustered with $200$ randomly generated feasible transformation matrices $A$ for the values $\epsilon = 0.004$, $\delta = 0$.} % in order to compare some important parameters
\end{figure}

Need: $3$ dominant Schur vectors $X$. Generate some feasible matrices $A$. Compute real rebinding  $\det(\Sreal) = \det(D\inv A^TA)$.

Assumption: since Schur vectors $X$ are the same as for the reversible case (matrix $P$ constructed in terms of this set of real, orthogonal vectors), the set of feasible matrices remains the same and thus, the computed real rebinding effect should also remain the same.

However, the \textbf{minimal} rebinding could change, since the optimization problem required the clustered Schur vectors as input. They are not necessarily the same as in the reversible case.
Maybe this estimation gets better or worse than it was the case for the reversible process?

Have given the process as $P_c = A\inv \Lambda_s A$, hence the columns of $A$ are multiples of the dominant Schur vectors.

All other things to compute remain the same. $S = D\inv A^T A$ and minimized rebinding effect for $\det(S)$ maximized.

In order to compute Schur Decomposition of the clustered process $Q_c$, we need to come back to the trick presented in section \ref{sec:nonreversible} to employ the symmetrized matrix $D^{1/2} P D^{-1/2}$. Thus we need at first to compute the stationary distribution of the clustered process.

%obtain a symmetric

\subsubsection*{$P$ is non-diagonalizable}

The Schur decomposition has an upper diagonal element, while the eigenvalue $0.99$ occurs algebraically twice.
\begin{figure}[h]
	%	\label{fig:reb_example_nonrev}
	\centering
	\subfigure[The minimal rebinding effect compared to the degree of non-reversibility of the clustered system $Q_c$.]{\includegraphics[width=0.31\textwidth]{figures/rebinding/200_random_schur/epsilon0.004delta0.01/reb_opt_nonrev_200_random.eps}}
	\hspace{1pt}
	\subfigure[The minimal and the real rebinding effect compared to the degree of non-reversibility of the clustered system $Q_c$.]{\includegraphics[width=0.31\textwidth]{figures/rebinding/200_random_schur/epsilon0.004delta0.01/reb_nonrev_200_random.eps}}
	%, increasing the probability of a fast \textbf{rebinding}.
	\hspace{1pt}
	\subfigure[The minimal rebinding effect $\det(\Sopt)$ compared to the real rebinding effect $\det(\Sreal)$ included in $Q_c$.]{\includegraphics[width=0.31\textwidth]{figures/rebinding/200_random_schur/epsilon0.004delta0.01/reb_200_random.eps}}
	\caption{The system described by the transition matrix $P$ is clustered with $200$ randomly generated feasible transformation matrices $A$ for the values $\epsilon = 0.004$, $\delta = 0.01$.} % in order to compare some important parameters
\end{figure}

\subsubsection*{$P$ is non-reversible with complex eigenvalues}

Complex eigenvalues always occur in pairs and are indicated by a complete $2 \times 2$-block in the Schur decomposition. For $\epsilon = 0.004$, $\delta = 0.01$ and $\gamma = 10^{-15}$, we get two complex eigenvalues $0.99+2.3 \cdot 10^{-9}i$ and $0.99-2.3 \cdot 10^{-9}i$.
\begin{figure}[h]
	%	\label{fig:reb_example_nonrev}
	\centering
	\subfigure[The minimal rebinding effect compared to the degree of non-reversibility of the clustered system $Q_c$.]{\includegraphics[width=0.31\textwidth]{figures/rebinding/200_random_schur/epsilon0.002gamma0.01/reb_opt_nonrev_200_random.eps}}
	\hspace{1pt}
	\subfigure[The minimal and the real rebinding effect compared to the degree of non-reversibility of the clustered system $Q_c$.]{\includegraphics[width=0.31\textwidth]{figures/rebinding/200_random_schur/epsilon0.002gamma0.01/reb_nonrev_200_random.eps}}
	%, increasing the probability of a fast \textbf{rebinding}.
	\hspace{1pt}
	\subfigure[The minimal rebinding effect $\det(\Sopt)$ compared to the real rebinding effect $\det(\Sreal)$ included in $Q_c$.]{\includegraphics[width=0.31\textwidth]{figures/rebinding/200_random_schur/epsilon0.002gamma0.01/reb_200_random.eps}}
	\caption{The system described by the transition matrix $P$ is clustered with $200$ randomly generated feasible transformation matrices $A$ for the values $\epsilon = 0.002$, $\gamma = 0.01$.} % in order to compare some important parameters
\end{figure}

%Overview of the different rebindings by Schur:
%\clearpage
%\begin{figure}[h]
%	\label{fig:reb_example_nonrev}
	\centering
	\subfigure[The minimal rebinding effect compared to the degree of non-reversibility of the clustered system $Q_c$.]{\includegraphics[width=0.31\textwidth]{figures/rebinding/200_random_schur/reb_opt_nonrev_200_random.eps}}
	\hspace{1pt}
	\subfigure[The minimal and the real rebinding effect compared to the degree of non-reversibility of the clustered system $Q_c$.]{\includegraphics[width=0.31\textwidth]{figures/rebinding/200_random_schur/reb_nonrev_200_random.eps}}
	%, increasing the probability of a fast \textbf{rebinding}.
	\hspace{1pt}
	\subfigure[The minimal rebinding effect $\det(\Sopt)$ compared to the real rebinding effect $\det(\Sreal)$ included in $Q_c$.]{\includegraphics[width=0.31\textwidth]{figures/rebinding/200_random_schur/reb_200_random.eps}}
	%\caption{The system described by the transition matrix $P$ is clustered with $200$ randomly generated feasible transformation matrices $A$ for the values $\epsilon = 0$, $\delta = 0$.} % in order to compare some important parameters
	\hspace{20pt}

	\subfigure[The minimal rebinding effect compared to the degree of non-reversibility of the clustered system $Q_c$.]{\includegraphics[width=0.31\textwidth]{figures/rebinding/200_random_schur/epsilon0.004/reb_opt_nonrev_200_random.eps}}
	\hspace{1pt}
	\subfigure[The minimal and the real rebinding effect compared to the degree of non-reversibility of the clustered system $Q_c$.]{\includegraphics[width=0.31\textwidth]{figures/rebinding/200_random_schur/epsilon0.004/reb_nonrev_200_random.eps}}
	%, increasing the probability of a fast \textbf{rebinding}.
	\hspace{1pt}
	\subfigure[The minimal rebinding effect $\det(\Sopt)$ compared to the real rebinding effect $\det(\Sreal)$ included in $Q_c$.]{\includegraphics[width=0.31\textwidth]{figures/rebinding/200_random_schur/epsilon0.004/reb_200_random.eps}}
	%\caption{The system described by the transition matrix $P$ is clustered with $200$ randomly generated feasible transformation matrices $A$ for the values $\epsilon = 0.004$, $\delta = 0$.} % in order to compare some important parameters
	\hspace{20pt}
	
	\subfigure[The minimal rebinding effect compared to the degree of non-reversibility of the clustered system $Q_c$.]{\includegraphics[width=0.31\textwidth]{figures/rebinding/200_random_schur/epsilon0.004delta0.01/reb_opt_nonrev_200_random.eps}}
	\hspace{1pt}
	\subfigure[The minimal and the real rebinding effect compared to the degree of non-reversibility of the clustered system $Q_c$.]{\includegraphics[width=0.31\textwidth]{figures/rebinding/200_random_schur/epsilon0.004delta0.01/reb_nonrev_200_random.eps}}
	%, increasing the probability of a fast \textbf{rebinding}.
	\hspace{1pt}
	\subfigure[The minimal rebinding effect $\det(\Sopt)$ compared to the real rebinding effect $\det(\Sreal)$ included in $Q_c$.]{\includegraphics[width=0.31\textwidth]{figures/rebinding/200_random_schur/epsilon0.004delta0.01/reb_200_random.eps}}
	%\caption{The system described by the transition matrix $P$ is clustered with $200$ randomly generated feasible transformation matrices $A$ for the values $\epsilon = 0.004$, $\delta = 0.01$.} % in order to compare some important parameters
	\hspace{20pt}
	
	\subfigure[The minimal rebinding effect compared to the degree of non-reversibility of the clustered system $Q_c$.]{\includegraphics[width=0.31\textwidth]{figures/rebinding/200_random_schur/epsilon0.002gamma0.01/reb_opt_nonrev_200_random.eps}}
	\hspace{1pt}
	\subfigure[The minimal and the real rebinding effect compared to the degree of non-reversibility of the clustered system $Q_c$.]{\includegraphics[width=0.31\textwidth]{figures/rebinding/200_random_schur/epsilon0.002gamma0.01/reb_nonrev_200_random.eps}}
	%, increasing the probability of a fast \textbf{rebinding}.
	\hspace{1pt}
	\subfigure[The minimal rebinding effect $\det(\Sopt)$ compared to the real rebinding effect $\det(\Sreal)$ included in $Q_c$.]{\includegraphics[width=0.31\textwidth]{figures/rebinding/200_random_schur/epsilon0.002gamma0.01/reb_200_random.eps}}
	%\caption{The system described by the transition matrix $P$ is clustered with $200$ randomly generated feasible transformation matrices $A$ for the values $\epsilon = 0.002$, $\gamma = 0.01$.} % in order to compare some important parameters
\end{figure}

\subsubsection*{Conclusion}

It makes sense that the solution of optimization problem bla using/based on Schur vectors is a \textbf{worse} estimation for the real rebinding effect than the solution based on eigenvectors.
When computing the minimal rebinding effect in terms of Schur vectors, we do not specify if the original process has been reversible or non-reversible.
Hence, this extended approach includes \textbf{more} possible solutions.
\clearpage

%If we examine a given clustered system $Q_c$, det(S_opt_eigen) \leq det(S_opt_schur)

%\begin{figure}[!ht]
%	\label{fig:reb_example_nonrev}
%	\centering
%	\subfigure[The minimal rebinding effect compared to the degree of non-reversibility of the clustered system $Q_c$.]{\includegraphics[width=0.46\textwidth]{figures/rebinding/200_random_nonrev/reb_opt_nonrev_200_random.eps}}
%	\subfigure[The minimal and the real rebinding effect compared to the degree of non-reversibility of the clustered system $Q_c$.]{\includegraphics[width=0.46\textwidth]{figures/rebinding/200_random_nonrev/reb_nonrev_200_random.eps}}
	%, increasing the probability of a fast \textbf{rebinding}.
%	\hspace{20pt}
%	\subfigure[The minimal rebinding effect $\det(\Sopt)$ compared to the real rebinding effect $\det(\Sreal)$ included in %$Q_c$.]{\includegraphics[width=0.46\textwidth]{figures/rebinding/200_random_nonrev/reb_200_random.eps}}
%	\caption{The system described by the transition matrix $P$ is clustered with $200$ randomly generated feasible transformation matrices $A$ for the values $\epsilon = 0.4$, $\delta = 0.7$.} % in order to compare some important parameters
%\end{figure}
%\newpage

%\subsubsection*{$Q$ has real eigenvalues}


%In order to find out the role of nonreversibility to $S_\textrm{opt}$, we begin to examine a reversible transition rate matrix $Q_\epsilon$. It contains a parameter $\epsilon = 0$, guaranteeing its reversibility. In small time steps, we increase $\epsilon$ and thereby increase the nonreversibility of the process, measured by $\Vert DQ - Q^TD \Vert$, where $D = \textrm{diag}(\mu_1, \dots, \mu_n)$ is the diagonal matrix consisting of the entries of the stationary distribution, compare characterization \eqref{eq:detailed} of reversibility.
%\marginpar{nop. need dense matrix}
%\eqref{}.

%\begin{equation*}
%	(Q_\epsilon)_{\epsilon = 0, \dots, 100} = 
%	\begin{pmatrix}
%		bla & \epsilon & 0 \\
%		0 & bla & 0 \\
%		0 & 0 & bla
%	\end{pmatrix}
%\end{equation*}