\section{Spectral Approach}
\label{sec:spectral}

In this section we will see that the spectrum of the transfer operator is highly connected to the metastability of the corresponding Markov process.
Namely, the number of metastable sets can be determined by the number of eigenvalues close to 1 and the corresponding eigenfunctions induce a metastable (full) decomposition.
%\marginpar{partition?}

%When a Markov model is reversible then its metastable sets can be found with the spectral approach.
%If the spectrum of the transition matrix $P$ has $m$ dominant eigenvalues, i.e.
%$m$ eigenvalues close to $1$, then there exists a decomposition of the state space
%of the process into $m$ metastable sets. (with high joint metastability)
% m dominant eigenvalues <-> m metastable sets

%The number of metastable sets/regions can be determined by the number of dominant (discrete!) %eigenvalues which are bounded away from the essential spectrum. So we have eigenvalues lambdai, % %corresponding eigenfunctions xi.

%metastable subsets: eigenvalues $\lambda_i \simeq 1$

\subsubsection*{Existence of dominant eigenvalues}
%\marginpar{maybe in chapter 2?}

%The \textit{multiplicity} of an eigenvalue $\lambda$ is defined as the dimension of the generalized %eigenspace. Eigenvalues of multiplicity $1$ are called  \textit{simple}.

Let us consider the transfer operator $\Pcal = \Pcal^t$ of a Markov process for some fixed $t$ in the Hilbert space $L_\mu^2(\X)$. \marginpar{ $L^2$, $L^1$?}
%\marginpar{$L^1$ Huisinga diss; why $L^2$, $L^1$ enough?}
%For further informations, see Kato\cite{kato1995}.
We are interested in \textit{dominant eigenvalues} of $\Pcal$, that is large eigenvalues which are close to 1 and separated from the rest of the spectrum.
The \textit{discrete spectrum} $\sigma_{\mathrm{discr}}(\Pcal)$ is the set consisting of all eigenvalues $\lambda \in \sigma(\Pcal)$ that are isolated and of finite multiplicity.
%contains all
The \textit{essential spectral radius} $r_{\mathrm{ess}}(\Pcal)$ is defined as
%follows
\begin{equation*}
r_{\mathrm{ess}}(\Pcal) = \inf \{ r \geq 0 \mid \lambda \in \sigma(\Pcal) \textrm{ with } |\lambda| > r \textrm{ implies } \lambda \in \sigma_{\mathrm{discr}}(\Pcal) \}.
\end{equation*}
The existence of dominant eigenvalues requires that the essential/continuous part of the spectrum is bounded away from the dominant elements of the discrete spectrum.
To ensure that the process we are considering actually possesses metastable sets, we need to pose some conditions on the spectrum of the transfer operator:
%determine some conditions

\begin{description}
    \item[C1] The essential spectral radius of $\Pcal$ is less than one; i.e. $r_{\mathrm{ess}} < 1$.
    \item[C2] The eigenvalue $\lambda=1$ of $\Pcal$ is simple and dominant; i.e. $\eta \in \sigma(\Pcal)$ with $|\eta| = 1$ implies $\eta = 1$.
\end{description}
We will not go into further details for which processes the two above conditions are fulfilled; 
%for which operators
some criteria about it can be found in Huisinga\cite{huisinga2001metastability}.
%for more informations about it see Huisinga\cite{huisinga2001metastability}.
Since these conditions are required for the later investigations, we will just assume that they are true.
%these properties

We need condition {\textbf{\textsf{C1}} to ensure that the continuous part of the spectrum is bounded away from the discrete eigenvalues. Otherwise they would not be dominant anymore and the process would be rather durchmischt than having any metastable sets.
%rapidly mixing?
Condition {\textbf{\textsf{C2}} however is important because a transfer operator with more than one eigenvalue of absolut value $1$ \marginpar{has periodic structures?} can be decomposed into stable/invariant sets, i.e. subsets which cannot be left. In that case we could just consider the different stable sets as independent processes. But that is not interesting for us. \marginpar{excludes modeling and interpretation problems} Instead, we want to know more about almost invariant sets and their critical/transition regions. \marginpar{C2 = ergodic?}
%We will consider only ergodic process which means theat the eigenvalue $1$ is unique.
\begin{thm}[{Sch\"utte\cite[theorem 4.16]{schutte2013metastability}}]
\label{thm:metastability_bounds}
%transfer operator for a process unique?
The transfer operator $\Pcal: L^2 \rightarrow L^2$ of a reversible process with properties \textrm{\textbf{\textsf{C1}}} and \textrm{\textbf{\textsf{C2}}} has the following spectrum:
 \marginpar{why discr. rechts?}
\begin{equation*}
\sigma(\Pcal) \subset [a,b] \cup \{\lambda_n\} \cup \dots \cup \{\lambda_2 \} \cup \{1\}
\end{equation*}
with $-1 < a \leq b < \lambda_n \leq \dots \leq \lambda_1 = 1$ and isolated, not necessarily simple eigenvalues of finite multiplicity that are counted according to multiplicity. 
\end{thm}
This theorem assures us the existence of a discrete set of dominant eigenvalues. In the following we will see that this property results in metastability.

%\subsubsection*{Optimal Decomposition}
\subsubsection*{Relation of dominant eigenvalues to metastable sets/Optimal Decomposition}

\begin{thm}[{Sch\"utte\cite[theorem 4.16]{schutte2013metastability}}]
\label{thm:metastability_bounds_1}
The metastability of an arbitrary decomposition $\Dcal = \{A_1,\dots, A_m\}$ of the state space $\X$ can be bounded from above by
\begin{equation*}
p(A_1,A_1), + \dots + p(A_m, A_m) \leq 1+ \lambda_2 + \dots + \lambda_m.
\end{equation*}
\end{thm}

Theorem \ref{thm:metastability_bounds_1} provokes the question if there exists an \textit{optimal} decomposition with highest possible metastability. \marginpar{maybe ill-conditioned}
%In fact, this problem might unfortunately be ill-conditioned.

The number of metastable sets is determined by the number of dominant eigenvalues.

\subsubsection*{Relation of dominant eigenfunctions to metastable decomposition}
\begin{thm}[Sch\"utte\cite{schutte2013metastability}]
Each single eigenfunction induces a metastable decomposition
\end{thm}
\begin{proof}
\end{proof}
The zeros of an eigenfunction (respectively change of sign) induce a metastable decomposition of the state space.
Different eigenfunctions result in different decompositions.
%(partition into sets??)
\marginpar{eigenfcts vs committor functions}
%result/ yield

%anschaulich
This theorem gives us a first demonstrative relation of the metastability of a process to its eigenvectors. This result is not yet optimal/very good. In the next sections, we will see that linear combinations of eigenvectors result in much better metastability.
\\

In figure \ref{fig:spectrum}, we get a good overview of that relation. We have a potential/energy landscape which has 4 energy minima, i.e. 4 regions where the process could be \textit{trapped} such that it is hard to get outside again. The transition matrix  shows this metastable behavious since we can see 4 regions which large probabilities to stay inside and very small probabilities to go to a different region.
Furthermore we see that the process has 4 dominant eigenvalues. Three of them have a change of sign, which induces the metastable decomposition.

\begin{figure}[!ht]
	\label{fig:spectrum}
	\centering
	\includegraphics[width=0.7\textwidth]{spectrum.jpg} %70% der Textbreite
	\caption{Relation of eigenvalues and eigenvectors to metastability of a process}
\end{figure}

\subsubsection*{Disadvantages}
%Advantages, Alternatives

So the spectral approach is suitable/convenient to characterize metastability of Markov processes.
But there are two disadvantages.
1: the result is only appliable on reversible processes, because real eigenvalues are only guaranteed if the transfer operator is self-adjoint. \marginpar{see}
2: eigenvector problem of the transfer operator has only global solutions.

Most of all, the previous approach in computing a metastable decomposition doesn't include/consider the transition regions of the process, so we need to refine/improve it.
