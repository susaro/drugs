\section*{Thesis Structure}

\subsubsection*{Chapter 1 - Markov State Models}
%mainly introduce notation
We give a short overview about Markov processes and how their evolution in time can be described by transition functions and transfer operators. %evolution of trajectories/densities
We show how such a continuous operator can be projected onto a finite-dimensional space with the aid of a Galerkin discretization.
Finally, we analyze the discretization error and the possible loss of the Markov property which can occur by this projection.
%This so called Recrossing effect will appear at the end of the thesis in molecular context and main topic of this thesis!

\subsubsection*{Chapter 2 - Dominant Structures}
In order to create a suitable Markov State Model preserving the long-time behaviour of the original process, we introduce the concept of metastability.
We define metastable sets mathematically and explain their relevance for molecular systems.
%Furthermore, 
We reveal their relation to the spectrum of the transfer operator and show that the ``best'' metastable decomposition is achieved in terms of fuzzy membership functions, which may be overlapping.
Finally, we extend this well-established clustering method to non-reversible processes by employing the Schur decomposition. %method/procedure

\subsubsection*{Chapter 3 - Rebinding Effect}
%main topic, principal topic
In this chapter, we characterize the rebinding effect as a memory effect which occurs in the context of receptor-ligand systems, a special case of molecular systems. %introduce
%This effect occurs by a projection
%In this chapter, we introduce the crucial point of this thesis, the Rebinding Effect.
%molecular dynamic/kinetic?
%We describe this effect in the context of a receptor-ligand-system, a special case of a molecular system, and set this effect in relation to the Recrossing Effect from chapter 1.
We apply the methods presented in the first chapters in order to rigorously describe a molecular system respectively its projection onto a finite subspace.
Finally, we compute a minimal bound for the rebinding effect as the solution of an optimization problem, for reversible as well as for non-reversible systems.

\subsubsection*{Chapter 4 - Illustrative Examples}

The results from chapter \ref{chap:rebinding} are verified by means of some illustrative examples.
At first, the minimal rebinding effect is computed for some clusterings of a reversible system in order to evaluate the quality of this estimation.
%At first, some clusterings of a reversible system are realized in order to evaluate the quality of the estimation for the minimal rebinding effect. %compare
%a reversible system is clustered
Afterwards, the same is done for a non-reversible system and compared to the outcome of the reversible case.
As the rebinding effect is characterized within receptor-ligand systems, such a system is presented.
In order to demonstrate the appearance of this effect in different contexts as well, we explain it in a system describing the chemical reaction of formic acid dimer. %how it is included in a system describing a chemical reaction
%existence/appearence

%that this effect appears also in other 