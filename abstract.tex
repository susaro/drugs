%\fbox{\parbox[c]{14.5cm}{
%``Examining the rebinding effect, which occurs when projecting a Markov process onto a finite state space and may spoil the Markovianity.''\\

\paragraph*{\centerline{Abstract}} \hspace{0pt} \\ %65 Wörter

%This thesis examines.. examine/investigate
The aim of this thesis is to investigate the rebinding effect, which occurs when projecting a Markov process onto a finite state space and may spoil the Markovianity.
%Based on the assumption
%Under the assumption of a
%Assuming
Under the assumption of a a fuzzy clustering in terms of membership functions $\chi = XA$, being a linear combination of Schur vectors,
a minimal bound for the rebinding effect included in a given system is computed as the solution of an optimization problem. %kinetics
%linear combination of eigenvectors or Schur vectors
%we investigate the minimal rebinding effect included in a given system.
%derived/deduced as the solution of an optimization problem
%}}\\[2ex]

%Rebinding	= Shortly after a dissociation, the probability for associating again (=rebinding) is still increased
%			= stabilizes the system; increases binding affinity