\section{Dominant Cycles}
%(non reversible NESS process)
\label{sec:nonreversible}

When it comes to computing a metastable decomposition for a nonreversible process, we have to face the problem that the eigenvalues/eigenvectors might be complex valued and thus, PCCA+ is not applicable. \marginpar{of transf.op.?}
One possibility/alternative/way to circumvent this problem is to consider the real Schur decomposition of the matrix(?) instead of its spectral decomposition. Then, we can apply PCCA+ to the real Schur vectors of the matrix instead to its eigenvectors.
This approach is feasible/possible, since the real Schur vectors span the same subspace as the corresponding complex Schur vectors and those span the same subspace as the corresponding eigenvectors. \marginpar{see ...}

%In this section, we give an overview about a special case of nonreversible processes (NESS); nonreversible but having a steady state/stat. dist./stat. meas.

%In addition, additionally
Furthermore, a nonreversible process can contain other dominant structures than just \textit{metastable sets/conformations}. It can as well contain \textit{metastable cycles}, that is subsets with a cyclic behaviour and a high probability to stay inside of such a cycle for a long time.
%structures/subsets

%In general, there exist two possible ways to uniquely describe a Markov chain; via a transition matrix or via cycles.

\subsubsection*{NESS processes}

%We begin with introducing a certain class of processes that are nonreversible, but still have some nice properties.

\begin{defi}[NESS process]
A Markov process is called \textit{nonequilibrium steady state (NESS)} process if it is nonreversible, but still has a steady state, \marginpar{what is that?} given by an invariant measure $\mu$ w.r.t. which the process is ergodic.
\end{defi}

$p_\tau(A,B) = p(\tau,A,B) =  \Prob(X_\tau = B \mid X_0 = A)$

As a NESS process is nonreversible, there are regions where the detailed balance equation is not fulfilled, i.e. there is an effective probability flow $p(\tau, A, B) - p(\tau,B,A) \neq 0$ between some subsets $A,B \subset S$ of the state space.

\subsubsection*{Flow of a process}
\marginpar{why only finite?}

In the following we consider an irreducible and aperiodic (i.e. ergodic) Markov chain on the finite state space $S= \{1,\dots,n\}$ given by the transition matrix $P$. Since this Markov chain is irreducible and aperiodic, it possesses a unique invariant measure $\mu$ that is positive everywhere. \marginpar{see ..}
Then $\mu$ is the normalized eigenvector of $P$ for the unique eigenvalue $\lambda = 1$.

\begin{defi}[Flow Matrix]
The probability flow associated to a Markov process is given by the flow matrix
\begin{equation*}
F = DP,
\end{equation*}
where $P$ is the transition matrix of the process and $D$ the diagonal matrix $D_{ii} = \mu_i$ with the entries of the invariant measure $\mu$.
\end{defi}
So the (steady state) probability flow from state $i$ to $j$ is given by $F_{ij} = \mu_i P_{ij}$.
If the process is reversible, the flow matrix $F$ is symmetric due to the detailed balance equation. For a NESS process, $F$ is not symmetric since there are states $i,j \in S$ with $F_{ij} \neq F_{ji}$.

\subsubsection*{Cycle Decomposition}

This flow must be decomposable into elementary cycles. \marginpar{why?}

\begin{defi}[Cycle of a process]
A $k$-cycle $\gamma$ on $S$ is an ordered sequence (up to cyclic permutations) of $k$ connected states $\gamma = (i_1,\dots,i_k)$ with length $|\gamma| = k$, i.e. the probability to get to the next state is always positive: $\Prob_{i_j, i_{j+1}} > 0$ and $\Prob_{i_k,i_1} > 0$.
\marginpar{1-step prob.?}
Cycles without repetition/self-intersections are called simple cycles. The set of all simple cycles is denoted by $\Ccal$.
\end{defi}

We want to make a cycle decomposition of the flow $F$, see Kalpazidou\cite{kalpazidou2007}.

\begin{defi}[Cycle/flow Decomposition]
A collection $\Ccal_+ \subset \Ccal$ of cycles $\gamma$ with real positive weights $w(\gamma)$ is a flow decomposition if for every edge $(i,j) \in S^2$ we have
\begin{equation*}
F_{ij} = \sum_{\gamma \supset (i,j)} w(\gamma),
\end{equation*}
where $(i,j) \subset \gamma$ if the edge $(i,j)$ is in $\gamma$.
\end{defi}

In order to make sense in a probabilistic context, we define the weight $w$ of a cycle $\gamma$ in the following way. Given a (realization of?) Markov chain $(X_t)_{t \in \T}$, we count the number of times $N_T^\gamma$ the process passes through a cycle $\gamma$ up to time $T$.
%w will be the mean/average number of the appearances of c along almost all the sample paths

\begin{defi}[Weight of a Cycle]
\begin{equation*}
w(\gamma) = \lim_{T \rightarrow \infty} \frac{N_T^\gamma}{T}.
\end{equation*}
\end{defi}
Since we are considering/assuming an ergodic process, this limit exists a.s. \marginpar{jian qian}


\subsubsection*{Dominant cycles/sets}
%[djur] chap 5
%same as PCCA+ but for nonreversible processes, so with Schur vectors instead of eigenvectors

We will see that dominant cycles have similar properties as dominant sets for reversible processes, i.e. large eigenvalues with $|\lambda| \approx 1$. But now the eigenvalues are lying in the complex plane and might be non-real (pairs of complex eigenv.).

%What is a metastable cycle?
\begin{defi}[Dominant Cycle]
= metastable cycle?
\end{defi}

So a cycle is dominant if there is a high probability inside the Markov chain to follow this cycle.

%What is the dominant structure of a nonreversible process? ibland cycle, ibland set?
%What is egentligen the problem with nonreversible processes? non-real eigenvalues

Dominant structures will be defined utilizing the dominant Schur vectors
of the transition matrix instead of its eigenvectors.
A membership matrix can be defined as a linear combination of these leading Schur vectors (spectral clustering with PCCA+).

%\subsubsection*{Spectrum of a NESS process}

\subsubsection*{Schur Decomposition}

We have the same situation/aim as in the previous sections: we have a Markov process on a large state space $S$ and we want to decompose it into a smaller state space consisting of clusters that belong to dominant structures of the process.

%Schur Decomposition = Matrix Decomposition

\begin{defi}[Schur Decomposition]
Let $P \in \mathbb{R}^{n \times n}$ be a transition matrix. Then it can be written as
\begin{equation*}
P = XRX^{-1},
\end{equation*}
where $X$ is a unitary matrix and $U$ is an upper triangular matrix, which is called a \textit{Schur form/matrix/decomposition} of $P$.
\end{defi}

Since $R$ is similar to $P$, both matrices have the same spectrum. Since $R$ is triangular, their eigenvalues are the diagonal entries of $R$.

A Schur Decomposition is not unique. \marginpar{...}
As $P$ is a real matrix, its non-real eigenvalues come in complex conjugate pairs. This fact can be used to build a \textit{real Schur form}, where $X$ and $R$ are both real matrices. But then $R$ is no longer triagonal, but only \textit{quasi-triangular}, allowing $2 \times 2$-blocks on its diagonal. \marginpar{picture}
The eigenvalues of the $2\times 2$-blocks are exactly the complex conjugate eigenpairs of $P$. \marginpar{see ..}

\begin{thm}[Real Schur Decomposition {\cite[Exc. I.3.24]{stewart1990matrix}}]
%Röblitz p. 22
If $A \in \R^{N \times N}$, then there exists an orthogonal matrix $U \in \R^{N \times N}$ such that
\begin{equation*}
U^TAU = T,
\end{equation*}
where $T$ is block-triangular with $1 \times 1$ and $2 \times 2$-blocks on its diagonal. The $1 \times 1$-blocks contain the real eigenvalues of $A$ and the eigenvalues of the $2 \times 2$-blocks are the complex eigenvalues of $A$.
\end{thm}

\begin{defi}[Schur Vector]
Let $\widetilde{R} \in \mathbb{R}^{m \times m}$ be a submatrix of $R$ (top left part of $R$). Then
\begin{equation*}
P = \xtilde \widetilde{R} \xtilde^{-1},
\end{equation*}
where $\xtilde \in \R^{n \times m}$ consists of the first $m$ columns of $X$. These vectors will be denoted as the dominant Schur Vectors of $P$. \marginpar{why?} Schur Values?
\end{defi}

\subsubsection*{Using PCCA+}
Djurdjevac Conrad et al\cite{djur2016} propose an algorithm in order to determine/get the desired membership vectors $\chi$.
\begin{enumerate}
\item Compute a real Schur decomposition $(\xtilde, R)$ of ...
\item Sort the Schur values and the $2 \times 2$-blocks such that they are in a descending order
%according to their maximal absolut values
\item Determine the submatrix $\xhat$ and solve PCCA+ equation (...) in order to get the membership functions $\chi$
\end{enumerate}

\subsubsection*{Computation of metastable cycles/sets}

%I. Beckenbach, L. Eifler, K. Fackeldey, A. Gleixner, A. Grever, M. Weber, J. Witzig: Mixed-Integer Programming for Cycle Detection in Non-reversible Markov Processes(2017)
Fackeldey and Weber\cite{fackeldey2017gen}
%GenPCCA+
%M. Weber and K. Fackeldey.   G-PCCA: Spectral clustering for non-
%reversible markov chains. ZIB-Report 15-35, Zuse Institute Berlin, 2015