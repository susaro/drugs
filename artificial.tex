We want to apply the results from chapter \ref{chap:rebinding} on some easy examples.

\section{A Transition Network Graph}

\section{An artificial (bivalent) binding Process}

One can distinguish between a monovalent binding process and a multivalent binding process (see ...), where multivalent processes are often considered as having a better binding affinity (see..).

For the monovalent case, the mathematical modeling of its kinetics is well understood.
....

Whenever the receptor molecules are spatially preorganized, the corresponding binding process is denoted as multivalent.

(especialle bivalent or polyvalent case often observed in nature)
These systems are of significant interest for pharmaceutical and technical applications. If the ligands are linked to each other in an appropriate way to match the preorganized receptor molecules and, thus, are also presented multivalently, then extremely high binding affinities are often observed.

So we consider here a bivalent process, as the the easiest multivalent case.

\begin{table}[ht]
\begin{tabular}{lccc}
\toprule
conformation              & statistical weight & holding probability & life time (ps) \\
\midrule
1    & 0.0001   & 0.9833 & 4.76   \\
2     & 0.0003  & 0.9666 & 2.34  \\
3     & 0.0041  & 0.9713 &    \\
4     & 0.0056 & 0.9987 &   \\
\bottomrule
\end{tabular}
\caption{relation of statistical weights to holding probabilities of the conformations}
\end{table}