With the aim of consolidating and illustrating the results from chapter \ref{chap:rebinding}, we apply them on some easy examples.
At first, we examine a \textbf{reversible} process and demonstrate the relation of the minimal rebinding effect to the degree of non-reversibility of the clustered system.
Afterwards, we repeatedly analyze the first example, now introducing small perturbations to make it \textbf{non-reversible}, in order to observe possible consequences for the rebinding effect.
%perturbations to non-reversibility. changes/consequences/differences
Since the rebinding effect was introduced by its occurrence in ligand-binding processes, we present an easy \textbf{bivalent binding system} and investigate the included rebinding effect. %introduced/motivated. occurence/influence
%Furthermore, Finally we come back
As a further application, we analyze a system describing the movement of \textbf{electron densities} in the chemical reaction of a molecule. %a system/matrix/molecule.
This example illustrates that the rebinding effect plays a role in many different kind of systems. It is not solely restricted to binding processes and consequently, for each process one has to interpret the meaning of this effect. %role of this effect. rebinding/recrossing. determine/interpret
%a possible impact
%see how that changes the rebinding effect. %Schur Decomposition
%has a different meaning 

In the following examples, the minimal rebinding effect is computed by solving optimization problem \eqref{eq:optimization} \marginpar{Schur} implemented with the help of Optimization Toolbox in Matlab.


%\section{Transition Network Graph}
%Relation of Rebinding Effect to Transition Regions
%change transition rates? rates of transitions b/w the metastable sets in order to identify the meaning of transition regions for the rebinding effect

%As a first easy example, we apply the results of the last chapter to an artificial system consisting of three metastable sets.

%\section{Stability}

%Have process by transition matrix $P$, cluster it with different membership functions $\chi_i$.
%\begin{itemize}
%	\item crisp membership functions: almost characteristic. $P_c = S\inv T \approx T$. $T$ very metastable -> system very stable, $F$ low
%	\item soft membership functions: $P_c = S\inv T$ not close to $T$. $T$ nov very metastable -> though system rather stable, $F$ low, by rebinding
%\end{itemize}


%\section{Rebinding Effect in an artificial (reversible) system}
%\section{Rebinding Effect in a reversible system}
%\section{Clusterings of a reversible system}
\section{Rebinding in a reversible System}
\label{sec:example_reversible}

The clustering $Q_c$ of a \textbf{reversible} process $Q$ can be \textbf{non-reversible}. Hence, we are interested to compare the minimal rebinding effect included in a clustered system with its degree of non-reversibility. Furthermore, we want to compare the \textbf{minimal} rebinding effect with the \textbf{real} rebinding effect stemming from the clustering. %actual clustering
%in order to know about/evaluate the quality of this estimation/bound

\subsubsection*{Different clusterings of a system}
%Cluster a system $Q$ with different tranformation matrices and compare the resulting rebinding effect vs nonreversibility

Let a \textbf{reversible} metastable process be given by the transition matrix
\begin{equation}
\label{eq:transition_matrix}
	P = 
	\begin{pmatrix}
		0.9876  &  0.0011  	&  0.0011  	&  0.0051	& 0.0051 	\\
		0.0033 	&  0.4973  	&  0.4949  	&  0.0036  	& 0.0009	\\
		0.0033  &  0.4949	&  0.4973  	&  0.0009  	& 0.0036	\\
		0.0076  &  0.0018 	&  0.0004	&  0.4969  	& 0.4932	\\
		0.0076  &  0.0004	&  0.0018   &  0.4932  	& 0.4969
	\end{pmatrix}.
\end{equation}
This matrix stems from Weber\cite{Weber2017} and obviously describes a system on three metastable sets, having the dominant eigenvalues $\sigma(P) = \{1, 0.99, 0.98\}$.
Thus, we are interested to examine different clusterings on a three-dimensional state space.
%In order to examine different clustered systems via $\chi = XA$, we generate $200$ random transformation matrices $A$.
%We want to examine different clustered systems out of this large system $P$.
%various/different/random
For that aim, we employ several transformation matrices $A \in \R^{3 \times 3}$, turning the dominant eigenvectors $X \in \R^{5 \times 3}$ into membership functions $\chi \in \R^{5 \times 3}$. In order that $\chi$ fulfills the partition of unity and non-negativity properties, the set of feasible matrices $F_A$ has to meet certain conditions, see Weber\cite[Chapter 3.4]{weber2006meshless}.
%Such matrices can be obtained by a function from PCCA+. \marginpar{want: crisper clustering}
Accordingly, we generate $200$ feasible transformation matrices $A$ and examine the rebinding effect caused by the projection.

\begin{figure}[!ht]
	\centering
	\subfigure[The minimal rebinding effect compared to the degree of non-reversibility of the clustered system $Q_c$.]{\includegraphics[width=0.46\textwidth]{figures/rebinding/200_random/reb_opt_nonrev_200_random.eps}}
	\subfigure[The minimal and the real rebinding effect compared to the degree of non-reversibility of the clustered system $Q_c$.]{\includegraphics[width=0.46\textwidth]{figures/rebinding/200_random/reb_nonrev_200_random.eps}}
	%, increasing the probability of a fast \textbf{rebinding}.
	\hspace{20pt}
	\subfigure[The minimal rebinding effect $\det(\Sopt)$ compared to the real rebinding effect $\det(\Sreal)$ included in $Q_c$.]{\includegraphics[width=0.46\textwidth]{figures/rebinding/200_random/reb_200_random.eps}}
	\caption{The system described by the transition matrix $P$ is clustered with $200$ randomly generated feasible transformation matrices $A$.} % in order to compare some important parameters
	\label{fig:reb_example}
\end{figure}

The results of this example are presented in figure \ref{fig:reb_example} and can be interpreted as follows.
In a), we see that the minimal rebinding effect correlates with the degree of non-reversibility $\Vert DQ_c - Q_c^T D \Vert_1$ of the clustered system, where $D = \diag(\pi_1, \dots, \pi_n)$ is the diagonal matrix consisting of the entries of the stationary distribution. %see detailed balance chap 1 
The higher this degree of non-reversibility, the higher the minimal rebinding effect.
This correlation also implies that for highly non-reversibly system, the minimal rebinding effect is a better estimation for the real rebinding effect, as represented in b).
The last picture shows that in general, $\det(\Sopt)$ can be a rather good or a rather bad estimation for the real rebinding effect.
The less rebinding is included in the system, the worse this estimation gets. %really?
\newpage






%\begin{table}[ht]
%\begin{tabular}{lccc}
%\toprule
%conformation              & statistical weight & holding probability & life time (ps) \\
%\midrule
%1    & 0.0001   & 0.9833 & 4.76   \\
%2     & 0.0003  & 0.9666 & 2.34  \\
%3     & 0.0041  & 0.9713 &    \\
%4     & 0.0056 & 0.9987 &   \\
%\bottomrule
%\end{tabular}
%\caption{relation of statistical weights to holding probabilities of the conformations}
%\end{table}

%\section{Quality of this lower bound}

%We project a process onto a smaller state space using PCCA+. Then we compare the real rebinding effect (provoked by he projection) to the minimal rebinding effect included in the projected process. Thereby we can see ``how good'' this lower bound is.

%\subparagraph*{Rebinding $\leftrightarrow$ Multivalence}

%\subparagraph*{Rebinding $\leftrightarrow$ Nonreversibility}

%\subparagraph*{Rebinding $\leftrightarrow$ Transition Regions}

%\subparagraph*{Rebinding $\leftrightarrow$ Application to different Processes (Raman)}